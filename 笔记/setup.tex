%================================
% note-setup-leftsidebox.tex
% fenglielie@qq.com 2025-09-12
%================================

\usepackage{amsmath,amsthm,amsfonts,amssymb}
\usepackage{mathtools}
\usepackage{mathrsfs}
\usepackage{bm}
\usepackage{extarrows}
\usepackage[a4paper, margin=1in]{geometry}
\usepackage{float}
\usepackage{indentfirst}
\usepackage{anyfontsize}
\usepackage{booktabs,multirow,multicol}
\usepackage[shortlabels,inline]{enumitem}
\usepackage{appendix}

\usepackage[dvipsnames]{xcolor}
\usepackage{graphicx}
\graphicspath{
    {./figure/}{./figures/}{./image/}{./images/}{./graphic/}{./graphics/}{./picture/}{./pictures/}
}
\usepackage{subcaption}

\usepackage[ruled,linesnumbered,noline]{algorithm2e}
\usepackage{listings}
\lstdefinestyle{simpleStyle}{
    basicstyle=\ttfamily\small,
    breaklines=true,
    keywordstyle=\color{blue},
    identifierstyle=\color{black},
    stringstyle=\color{violet},
    commentstyle=\color[RGB]{34,139,34},
    showstringspaces=false,
    numbers=left,
    numbersep=2em,
    numberstyle=\footnotesize,
    frame=single,
    framesep=1em,
}
\lstset{style=simpleStyle}

\usepackage{hyperref}
\hypersetup{
    colorlinks=true,linkcolor=,urlcolor=cyan
}

\renewcommand*{\proofname}{\normalfont\bfseries Proof}

\usepackage{thmtools}

%% define environments
\declaretheorem[style=plain, name=Theorem, numbered=yes, numberwithin=section]{theoremplain}
\declaretheorem[style=plain, name=Proposition, numbered=yes, sibling=theoremplain]{propositionplain}
\declaretheorem[style=plain, name=Corollary, numbered=yes, sibling=theoremplain]{corollaryplain}
\declaretheorem[style=plain, name=Definition, numbered=yes, sibling=theoremplain]{definitionplain}


\declaretheorem[style=plain, name=Theorem, numbered=yes, numberwithin=section]{theorem}
\declaretheorem[style=plain, name=Theorem, numbered=no]{theorem*}

\declaretheorem[style=plain, name=Proposition, numbered=yes, sibling=theorem]{proposition}
\declaretheorem[style=plain, name=Proposition, numbered=no]{proposition*}

\declaretheorem[style=plain, name=Corollary, numbered=yes, sibling=theorem]{corollary}
\declaretheorem[style=plain, name=Corollary, numbered=no]{corollary*}

\declaretheorem[style=plain, name=Lemma, numbered=yes, sibling=theorem]{lemma}
\declaretheorem[style=plain, name=Lemma, numbered=no]{lemma*}

\declaretheorem[style=plain, name=Claim, numbered=yes, sibling=theorem]{claim}
\declaretheorem[style=plain, name=Claim, numbered=no]{claim*}

\declaretheorem[style=definition, name=Definition, numbered=yes, numberwithin=section]{definition}
\declaretheorem[style=definition, name=Definition, numbered=no]{definition*}

\declaretheorem[style=definition, name=Example, numbered=yes, numberwithin=section]{example}
\declaretheorem[style=definition, name=Example, numbered=no]{example*}

\declaretheorem[style=definition, name=Problem, numbered=yes, numberwithin=section]{problem}
\declaretheorem[style=definition, name=Problem, numbered=no]{problem*}

\declaretheorem[style=remark, name=Remark, numbered=yes, numberwithin=section]{remark}
\declaretheorem[style=remark, name=Remark, numbered=no]{remark*}

\declaretheorem[style=remark, name=Note, numbered=yes, numberwithin=section]{note}
\declaretheorem[style=remark, name=Note, numbered=no]{note*}

\declaretheoremstyle[headfont=\bfseries, bodyfont=\normalfont, spaceabove=3pt, spacebelow=3pt, qed=\ensuremath{\square}]{solutionstyle}

\declaretheorem[style=solutionstyle, name=Solution, numbered=yes, numberwithin=section]{solution}
\declaretheorem[style=solutionstyle, name=Solution, numbered=no]{solution*}

\usepackage[most]{tcolorbox}

\newcommand{\newtcbenvironment}[2]{
    \tcolorboxenvironment{#1}{#2, enhanced, breakable, sharp corners,leftrule=2pt, rightrule=0pt, toprule=0pt, bottomrule=0pt}
    \tcolorboxenvironment{#1*}{#2, enhanced, breakable, rounded corners,leftrule=2pt, rightrule=0pt, toprule=0pt, bottomrule=0pt}
}

%% define styles

\newtcbenvironment{theorem}{colframe=RoyalPurple, colback=RoyalPurple!8}
\newtcbenvironment{proposition}{colframe=RoyalPurple, colback=RoyalPurple!8}
\newtcbenvironment{corollary}{colframe=NavyBlue, colback=SkyBlue!8}
\newtcbenvironment{lemma}{colframe=NavyBlue, colback=SkyBlue!8}
\newtcbenvironment{claim}{colframe=NavyBlue, colback=SkyBlue!8}

\newtcbenvironment{definition}{colframe=ForestGreen, colback=ForestGreen!5}
\newtcbenvironment{example}{colframe=RawSienna, colback=RawSienna!5}
\newtcbenvironment{problem}{colframe=WildStrawberry!30, colback=WildStrawberry!5}

%% cbox
\newtcolorbox{cbox}[1][]{%
    enhanced,
    breakable,
    sharp corners,
    leftrule=2pt, rightrule=0pt, toprule=0pt, bottomrule=0pt,
    colframe=SkyBlue,
    colback=SkyBlue!8,
    #1
}

%% cover
\usepackage{titling}
\newcommand{\extrainfo}{}
\renewcommand{\extrainfo}[1]{\renewcommand{\extrainfocontent}{#1}}
\newcommand{\extrainfocontent}{}
\newcommand{\makecover}[1]{%
    \begin{titlepage}
    \newgeometry{margin=0in}
    \parindent=0pt
    \includegraphics[width=\linewidth]{#1} % size = 1280*1024
    \vfill
    \begin{center}
        \parbox{0.618\textwidth}{%
            \raggedleft{\bfseries \Huge \thetitle} \\[0.6pt]
            \rule{0.618\textwidth}{4pt} \\
        }
    \end{center}
    \vfill
    \begin{center}
        \parbox{0.618\textwidth}{%
          \raggedleft\Large
            \begin{tabular}{r}
                \theauthor \\
                \thedate \\
            \end{tabular}%
        }
    \end{center}
    \vfill
    \begin{center}
        \parbox[t]{0.7\textwidth}{\centering \itshape \extrainfocontent}
    \end{center}
    \vfill
    \end{titlepage}
    \restoregeometry
    \thispagestyle{empty}
}
% USAGE
% \extrainfo{xxx}
% \makecover{/path/to/cover.png}

%记号定义
\newcommand{\nn}{\mathbb{N}}
\newcommand{\zz}{\mathbb{Z}}
\newcommand{\qq}{\mathbb{Q}}
\newcommand{\rr}{\mathbb{R}}
\newcommand{\cc}{\mathbb{C}}
\newcommand{\ff}{\mathbb{F}}
\newcommand{\ffp}{\mathbb{F}_p}
\newcommand{\Log}{\operatorname{Log}}