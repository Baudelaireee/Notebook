\section{Commutative ring}

\subsection{Classification}
There are many types of ring, we do a general classification here for a clear implication.

\begin{definition}
    Let \(R\) be a domain, and then we define

    - \(R\) is a \textbf{PID (principal ideal domain)} if all the ideal is princiapl.

    - \(R\) is a \textbf{UFD (Uique factorization domai)} if every non-zero and non-unit element can be wirtten as a finite product of irreducible elements, and the representation is unique up to the order: if \(p_i \) and \(q_j\) are all irreducible such that \(q_1 \cdots q_n = p_1 \cdots p_m\), then \(m=n\) and there exists a permutation \(\sigma \in S_n\) such that \(q_i = p_{\sigma(i)}\).

    - \(R\) is a \textbf{ED (Euclidean domain)} if a domain having a division algorithm: there exists a degree function \(N: R-\{0\} \to \n\) such that for all \(f,g \in R\) with \(f \neq 0\), we can conclude \(q, r \in R\) satisifying 
    \[g = qf +r\]
    where either \(r=0\) or \(N(r)<N(f)\)

    \begin{remark}
        We must notice the relationship of the definition and the arithemtic properties: 

\begin{center}
\begin{tabular}{|c|c|}
\hline
\textbf{Ring} & \textbf{Arithmetic Property of \(\z\)} \\
\hline
Euclidean Domain & Euclidean Algorithm \\
\hline
UFD & Fundamental Theorem of Arithmetic \\
\hline
PID & Bézout's Theorem \\
\hline
\end{tabular}
\end{center}
   \end{remark}
\end{definition}

So there exists a baisc implication:
\begin{proposition}
    Let \(R\) be a domain 
    \[\text{ED} \implies \text{PID} \implies \text{UFD} \]

    \begin{proof}
        \textbf{First implication:} Let \(R\) be an euclidean domain and \(I   \) be an non-zero ideal. Suppose that \(N\) is the degree function, then \(S = \{N(x) | x \in I-\{0\}\} \subset \n\), by the well-ordering principle of natural number, we can conclude a minimal element \(N(a) \in S\), then we will prove that \(I = (a)\). For any \(n \in I\) we can find \(r,s \in R\) such that \(n = sa+r\). Since \(I\) is an ideal, then \(sa \in I\) and so immediately \(r \in I\). Notice that \(N(r) < N(a)\), by the minimal of \(a\) we can conclude that \(r\) must be zero, so \(n \in (a)\), which implies \(I \subset (a)\), and another direction is trival, so we finish the proof.
        
        \textbf{Second implication:}
    \end{proof}
\end{proposition}




\subsection{Polynomial}


\subsubsection{arithmetic of polynoimal}
In this section we will discuss the arithemtic properties of polynoimal under a commutative ring \(R\), The proof of properties is nearly same with the proof of arithemtic properties of intger.

\begin{theorem}[Euclidean Algorithm] $ \\$
    Suppose that \(R\) is a domain and \(f,g \in R[X]\) with \(f\) a monic polynomial, then there exists unique polynoimals \(r,s \in R[X]\) such that
    \[g = sf + r\]
    with \(\deg r < deg f\) or \( \deg r =0\)
     
    \begin{proof}
        \textbf{Existence:}        
         If \( \deg g < \deg f \), simply set \( s = 0 \), \( r = g \). Then \( g = sf + r \) and \( \deg r < \deg f \), as required.  Suppose \( \deg g \geq \deg f \). Let
        \[
        g = a_n X^n + \dots + a_0, \quad f = X^d +...+b_0.
        \]
        Let \( n = \deg g \), \( d = \deg f \). Then define
        \[
        s_1 = a_nX^{n-d}
        \]
        Then the degree of \(r_1 = g -s_1f\) will be less than \(n\), if the degree is still larger than \(d\), we do the same procedure till some integer \(k\):
        \begin{align*}
            r_1 &= g-s_1f \\
            r_2 &= r_1 - s_2f\\
            &...\\
            r_k &= r_{k-1} - s_kf
        \end{align*}
        where \(s_i = a_iX^{m_i}\) with \(a_i\) the leading cofficient of \(r_{i-1}\) and \(m_i = \deg r_{i-1} -d\). \(k\) satisfies
        \[\deg r_1 > \deg r_2 > ... > \deg r_{k-1} \geq d=deg f > \deg r_{k}\]
        Finally, adding them togther we get 
        \[r = r_k, s = s_1+s_2+...+s_k\]
        \textbf{Uniqueness:} Suppose there exist two such decompositions:
        \[
        g = s_1 f + r_1 = s_2 f + r_2.
        \]
        Then:
        \[
        (s_1 - s_2)f = r_2 - r_1.
        \]
        The left-hand side is divisible by \( f \) if \(s_1 \neq s_2\), and the right-hand side has degree strictly less than \( \deg f \), unless it is zero. Since \( R \) is an integral domain, this implies \( r_1 = r_2 \) and thus \( s_1 = s_2 \). Hence the decomposition is unique.
        \end{proof}
\end{theorem}

\begin{remark}
    For a field \(K\), \(K[X]\) will be an excellent object to study (we call it \textbf{\(K\)-algebra}) since it satisfies Euclidean Algorithm, Furthermore it will satisfy all arithemtic properties in \(\z\), in brief, it is a Euclidean ring with degree as its euclidean function
\end{remark}

