\documentclass[en,geye,blue,pc,12pt]{elegantnote}

\usepackage{
amsmath,% AMS basic math stuff
amsthm,% AMS theorem defining stuff
amsfonts,% defines the blackboard bold fonts for \Z, \R, etc
}
%证明符号
\usepackage[all,cmtip]{xy}
\usepackage{amssymb}
\renewcommand\qedsymbol{$\blacksquare$}



\newcommand{\n}{\mathbb{N}}
\newcommand{\z}{\mathbb{Z}}



\title{Math Remark
\\ Algebraic Structure
}

\author{X}
\institute{Elegant\LaTeX{} Program}

\begin{document}

\maketitle

\newpage

\tableofcontents

\newpage

%-----------------------------------------------------
\section{Number System}
Without talking some basic knowledge of Mathmatics logic, we generally define the object we want to study: Number System is a set of "number" and equipped by certain opreations. Here "number" is not necessary a real number like 1,2,3 we face daily in caculation, later we will aware that "number is actually a represent of a system, or using the language of the category, a normal system like \(\n\) is just a represent object we choose in a category \(Cat(\n)\) (The collection of the system same as \(\n\)).

The main goal of this part is to construct the different number system begin from the natural number \(\n\), the procedure often can be found in the textbook and the exercise, and the extension of distinct system inspire us to define the new algebra object.

\subsection{From \texorpdfstring{$\n$}{TEXT} to \texorpdfstring{$\z$}{TEXT} }
The common idea is to add a new element \(-1\) to the system such that 
\[1+(-1)=0\]
which refers to the completion of the unit of \(\n\). We should notice that \(\z\) is a typical commutative ring with 1 identity, by comparison \((\n,+)\) is even not an abelian group, so by add a new element to the system we can clearly get the another "direction", which means \(\{0,-1,-2,....\}\) also forms a number system like \(\n\) loosely speaking. we can caulate that \(-2=(-1)+(-1)\) by 
\[2+(-1)+(-1)=1+1+(-1)+(-1)=0\]
so we can define that \(-k\) is the sum of \(k\) same number \(-1\). 

Here we reconsider the negative number from the inspiration of the substraction, we can know that
\[-1 = 1-2 = 2-3 = 3-4 = ...\]
and 
\[1 = 2-1 = 3-2 = 4-3 = ...\]
so we can define a binary relation on \(\n \times \n\) by 
\[(a,b) \sim (c,d) \Longleftrightarrow a+d = b+c\]
In fact we should notice that we want to write \(a-b =c-d\), but we do not still define substraction formally, so we do the change. we can define that the relation is equivalent, which is easily to verify:
\begin{itemize}
    \item \textbf{reflexivity:} \((a,b) \sim (a,b) \Longleftrightarrow a+b = b+a\).
    \item \textbf{Symmetry:} \[
\begin{array}{rl}
  (a, b) \sim (c, d) & \iff a + d = b + c  \\
&\iff c+b = b+c = a+d = d+a \\ 
&\iff (c,d) \sim (a,b) \end{array}
\]
    \item \textbf{transitivity:}
    \[\begin{array}{rl}
        (a, b) \sim (c, d) \wedge (c,d) \sim (e,f) & \iff a + d = b + c \wedge c+f =d+e \\
      &\iff a+(d+c)+f = b+(c+d)+e\\ 
      &\iff a+f = b+e \\
      &\iff (a,b) \sim (e,f) \end{array}
      \]
\end{itemize}

Hence we can use this equiavlence relation to construct the intger.
\begin{proposition}
    Suppose \(X = \n \times \n\), and we put \([a,b]\) to be the equiavlence class of the class containning \((a,b) \in X\), then following result can be verified:
    \\(1) the following operation is well-defined.
    \[[a,b]+[c,d] = [a+c,b+d]\]
    \[[a,b]\cdot[c,d] = [ac+bd, ad+bc]\]
    \\(2) the system \((X/\sim,+,\cdot)\) form a commutative ring with multiplicative identity.
    \\(3) The map \(f: \n \to \z, n \mapsto [n,0]\) is injective and additive
    \[f(n+m) = f(n) + f(m)\]
    \\(4) If \(X_+ = \{[n,0]: n\in \n\}\) and \(X_- = \{[0,n]: n\in \n\}\), then
    \[X/\sim = X_+ + X_-\]
\end{proposition}



\end{document}