\documentclass[en,geye,blue,pc,12pt]{elegantnote}

\usepackage{
amsmath,% AMS basic math stuff
amsthm,% AMS theorem defining stuff
amsfonts,% defines the blackboard bold fonts for \Z, \R, etc
}
%证明符号
\usepackage{bbm}
\usepackage[all,cmtip]{xy}
\usepackage{amssymb}
\renewcommand\qedsymbol{$\blacksquare$}



\newcommand{\rr}{\mathbb{R}}
\newcommand{\msp}{(X,\mathcal{E} ,\mu)}
\newcommand{\psp}{(\Omega ,\mathcal{T}   ,P)}
\newcommand{\cc}{\mathbb{C}}
\newcommand{\cha}{\mathbbm{1}}


\title{Math Remark
\\ Algebraic Structure
}

\author{X}
\institute{Elegant\LaTeX{} Program}

\begin{document}

\maketitle

\newpage

\tableofcontents

\newpage

%-----------------------------------------------------
\section{Number System}
Without talking some basic knowledge of Mathmatics logic, we generally define the object we want to study: Number System is a set of "number" and equipped by certain opreations. Here "number" is not necessary a real number like 1,2,3 we face daily in caculation, later we will aware that "number is actually a represent of a system, or using the language of the category, a normal system like \(\n\) is just a represent object we choose in a category \(Cat(\n)\) (The collection of the system same as \(\n\)).

The main goal of this part is to construct the different number system begin from the natural number \(\n\), the procedure often can be found in the textbook and the exercise, and the extension of distinct system inspire us to define the new algebra object.

\subsection{From \texorpdfstring{$\n$}{TEXT} to \texorpdfstring{$\z$}{TEXT} }
The common idea is to add a new element \(-1\) to the system such that 
\[1+(-1)=0\]
which refers to the completion of the unit of \(\n\). We should notice that \(\z\) is a typical commutative ring with 1 identity, by comparison \((\n,+)\) is even not an abelian group, so by add a new element to the system we can clearly get the another "direction", which means \(\{0,-1,-2,....\}\) also forms a number system like \(\n\) loosely speaking. we can caulate that \(-2=(-1)+(-1)\) by 
\[2+(-1)+(-1)=1+1+(-1)+(-1)=0\]
so we can define that \(-k\) is the sum of \(k\) same number \(-1\). 

Here we reconsider the negative number from the inspiration of the substraction, we can know that
\[-1 = 1-2 = 2-3 = 3-4 = ...\]
and 
\[1 = 2-1 = 3-2 = 4-3 = ...\]
so we can define a binary relation on \(\n \times \n\) by 
\[(a,b) \sim (c,d) \Longleftrightarrow a+d = b+c\]
In fact we should notice that we want to write \(a-b =c-d\), but we do not still define substraction formally, so we do the change. we can define that the relation is equivalent, which is easily to verify:
\begin{itemize}
    \item \textbf{reflexivity:} \((a,b) \sim (a,b) \Longleftrightarrow a+b = b+a\).
    \item \textbf{Symmetry:} \[
\begin{array}{rl}
  (a, b) \sim (c, d) & \iff a + d = b + c  \\
&\iff c+b = b+c = a+d = d+a \\ 
&\iff (c,d) \sim (a,b) \end{array}
\]
    \item \textbf{transitivity:}
    \[\begin{array}{rl}
        (a, b) \sim (c, d) \wedge (c,d) \sim (e,f) & \iff a + d = b + c \wedge c+f =d+e \\
      &\iff a+(d+c)+f = b+(c+d)+e\\ 
      &\iff a+f = b+e \\
      &\iff (a,b) \sim (e,f) \end{array}
      \]
\end{itemize}

Hence we can use this equiavlence relation to construct the intger.
\begin{proposition}
    Suppose \(X = \n \times \n\), and we put \([a,b]\) to be the equiavlence class of the class containning \((a,b) \in X\), then following result can be verified:
    \\(1) the following operation is well-defined.
    \[[a,b]+[c,d] = [a+c,b+d]\]
    \[[a,b]\cdot[c,d] = [ac+bd, ad+bc]\]
    \\(2) the system \((X/\sim,+,\cdot)\) form a commutative ring with multiplicative identity.
    \\(3) The map \(f: \n \to \z, n \mapsto [n,0]\) is injective and additives   
    \begin{equation*}
      f(n+m) = f(n) + f(m)
    \end{equation*}
    (4) If \(X_+ = \{[n,0]: n\in \n\}\) and \(X_- = \{[0,n]: n\in \n\}\), then
    \[X/\sim = X_+ + X_-\]
\end{proposition}

\begin{proof}
  (1) For any \((x,y) \in [a,b]\) and \((x',y') \in [c,d]\), we define the addition by
  \[(x,y)+(x',y') = (x+x',y+y')\]
  then we have 
  \[x+b=y+a \wedge x'+d = y'+c\]
  add them togther we get 
  \[(x+y)+(b+d) = (x'+y')+(a+c)\]
  which implies \((x+x',y+y') \sim (a+c,b+d)\), and then clearly \([a,b]+[c,d] \subset [a+c,b+d]\). The proof can be finished here because that the caculation will always be in the \([a+c,b+d]\), so we just need to check the corresponding caculation is really an injective then the definition will be well, but let us finish another direction, because it refers to the properties of natural number.

  Mutually, if \((x,y)\in [a+c,b+d]\), then we have 
  \[x+b+d = y+a+c\] 
  By the choice of the element in class \([a,b]\), we can always find \((i,j)\) such that \(i\leq a\) and \(j \leq b\), one thing should be pointed that we do not use substraction, usually we fix \(j\) such that \(y+j \geq x\), which can be ensured by Archimedean Property , i.e. \(\n\) is not bounded. and then \(i\) will be founded by \textbf{the properties of additive group}. hence here we can write down \[(x,y) = (i,j) + (x_i,y_j)\]
  we do not write \(x_i =x-i\) and \(y_j = y-j\) to prevent the abuse of the substraction. Finally, we can get two equation
  \[i+x_i+b+d = j+y_j+a+c\]
  \[i+b = j+a\]
  then we can use the cancellation law of the group to get 
  \((x_i,y_j) \sim (c,d)\), which finish our proof.
\end{proof}

\begin{remark}
  We finish our definition, and formally we define the integer is such a ring which is an quotient set,
  \[\z = \n \times \n / \sim\]
  and we denote \([0,0]\) by \(0\), denote \([1,0]\) by \(1\). And \([a,b]\) is called a positive number if \(a>b\), and we denote it by \(a-b\), which is the solution of \(a=b+x\), otherwise we call \([a,n]\) a negiative number with the notation \(-(a-b)\). More directly, if \(n \in \n - \{0\}\), we have the simialrly notation in \(\z\) by 
  \[ n = [n,0] \qquad -n = [0,n]\]
  then we will have some basic properties as following
  \begin{itemize}
    \item \(-(-a) = a\) or \(a+(-a)=0\)
    \item \((-a)b = a(-b) = -ab\)
  \end{itemize}
  By some verification, we can use these symbols to replace the equivalence class to refers the element in the integer ring, i.e. we finish the construction of the integer.

\end{remark}
\end{document}