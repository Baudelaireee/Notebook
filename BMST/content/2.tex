\section{p-adic analytic funnctions}

The main goal of this part is to completely solve the equation by p-adic methods, the concept of p-adic analytic functions, in particular the p-adic logarithm and exponential, together with Strassman's theorem, will be the key points.

By comparison to \(\cc\), it is easier to apply the properties of sequence in \(\cc_p\) by proposition 2.8 (note), similarly we can study the convergence of power series in a non-archimedean field by considering the radius convergence
\[R = 1/\limsup_{n \rightarrow \infty}\sqrt[n]{|a_n|}\]
since an ultrametric is still a metric, it shows a good properties as following:
\begin{proposition}
    Let \(f(X)=\sum_{n \geq 0}a_nX^n\) be a power series with coefficients in \(\cc_p\), let the radius of convergence be \(R>0\), then\\
    (1) \(f(x)\) converges if \(|x|<R\), and diverges if \(|x|>R\)\\
    (2) When \(|x|=R\), \(f(x)\) converges if and only if \(\lim_{n \rightarrow \infty}|a_n|R^n = 0\). \\
    (3) \(f\) is continous on the convergence domian \(D\).

    \begin{proof}
        Proof of (1) is same with the proof in \(\cc\). For any \(|x|=R\), the series \(\sum_{n \geq 0}a_nx^n\) converges if and only if sequence \((|a_nx^n|)_{n} = (|a_nR^n)_{n}\) converges by proposition 2.8. To prove (3) we take two points \(x,y\in D\) and let \(M = \max(|x|,|y|)\), then
        \begin{align*}
            |f(x)-f(y)| &= |\sum_{n \geq 0}a_n(x^{n-1}+x^{n-2}y+...+y^{n-1})(x-y)| \\
            &\leq |x-y|\max_{n\geq 0}{|a_nM^{n-1}|}
        \end{align*}
        Notice that \(f(x)\) and \(f(y)\) converge, then the sequence \((|a_nM^n|)_{n}\) converges to zero, so it will be bounded by a constant \(C\), which implies \(|f(x)-f(y)|\leq\frac{C}{M}|x-y|\), immediately \(f\) is continous.
    \end{proof}
\end{proposition}


\begin{definition}
    Let \(B_p(a,r) = \{x \in \cc_p| |x-a|_p <r \}\) be  the open ball of radius \(r\) around \(a\) in \(\cc_p\).

    -The p-adic logarithm is the p-adic analytic function \(\log_p: B_p(1,1) \to \cc_p\) defined by
    \[\log_p(x) := \sum_{n=1}^{\infty} (-1)^{n+1} \frac{(x-1)^n}{n}\]

    -The p-adic exponential is the p-adic analytic function \(\exp_p : B_p(0,p^{-1/(p-1)}) \to \cc_p\) defined by
    \[\exp_p (x) :=  \sum_{n=0}^{\infty} \frac{x^n}{n!}\]
\end{definition}

We will verify the statement is well-defined. For the reader who is familiar with p-adic analytic functions, this section can be skipped. From now on, unless stated otherwise, \(\log(\cdot)\) and \(\exp(\cdot)\) will denote the p-adic analytic function as above. \newline 

It is easier to check the statement for the p-adic logarithm by observing \(v_p(n) \leq \frac{\ln(n)}{\ln(p)}\) for any integer \(n \geq 1\), because any integer \(n \in [p^k,p^{k+1})\) has valuation at most \(k\), then for any \(|x-1|_p = p^{-r} < 1\) we have
\[ \lim_{n \rightarrow \infty} |1/n|_p |x-1|_p^n =\lim_{n \rightarrow \infty} p^{v_p(n)-nr} = p^{\lim_{n \rightarrow \infty}n(\frac{v_p(n)}{n}-r)} = 0\]
When \(|x-1|_p = 1\), notice that sequence \(|1/n|_p\) diverges, so the \(B_p(1,1)\) is the domain of convergence. Similarly,we will compute the radius of convergence of the p-adic exponential function.

\begin{lemma}
    Let \(n \in \n\) and \(S_n\) denotes the sum of the digits of \(n\) in base p, then 
    \[v_p(n!) =  \frac{n-S_n}{p-1}\]

    \begin{proof}
        Firstly we prove \(v_p(p^n!) = \frac{p^n-1}{p-1}\) for any positive integer \(n\) by recurrence. When \(n=1\), \(v_p(p) = 1\); for an intger \(n \geq 2\) we assume that \(v_p(p^{n-1}!) = \frac{p^{n-1}-1}{p-1}\), then \(p^n! = p^{n-1}! \cdot \prod_{k=1}^{p-1} A_k\) with
        \[A_k = (kp^{n-1}+1) \times (kp^{n-1}+2)\times \cdots \times (k+1)p^{n-1}\]
        then
        \begin{align*}
            v_p(p^n !) &= v_p(p^{n-1}!) + \sum_{k=1}^{p-1}v_p(A_k)\\
            &= v_p(p^{n-1}!) + (p-1)v_p(p^{n-1}!)+1 \\
            &= \frac{p^{n-1}-1}{p-1}+p^{n-1} \\
            &= \frac{p^{n}-1}{p-1}
        \end{align*}
        Hence we finish our recurrence. And if we tkae \(a \in {1,...,p-1}\), then the formula can be generalized, as the following shows:
        \begin{align*}
            v_p[(ap^n)!] &= \sum_{k=0}^{a-1}v_p[(k\cdot p^n +1)\times(k\cdot p^n +2) \times \cdots \times  (k\cdot p^n + p^n)] \\
            &= \sum_{k=0}^{a-1}v_p(p^n!) = a\frac{p^n-1}{p-1}
        \end{align*}
        Finally we prove the lemma by recurrence. Assuming that for any integer \(n-1\) the identity holds, and \(n= ap^r+m\) with \(a \in \{1,...,p-1\} \) and \(m<p^r<n-1\), then
        \begin{align*}
            v_p(n!) &= v_p(ap^r!)+\sum_{k=1}^{m}v_p(ap^r!+k)\\
            &= v_p(ap^r!)+v_p(m!) \\
            &= a\cdot \frac{p^r-1}{p-1}+\frac{m-S_m}{p-1}\\
            &= \frac{(ap^r+m)-(a+S_m)}{p-1} = \frac{n-S_n}{p-1}
        \end{align*}
    \end{proof}
\end{lemma}

By above lemma, the exponentials converges in given domain. For any \(|x|_p < p^{-1/p-1}\), there exists \(\epsilon>0\) such that \(\epsilon = (p-1)v_p(x)-1\), so we can estimate
\[v_p(x^n/n!) = \frac{n\epsilon-S_n}{p-1} \geq \frac{n\epsilon-p(ln(n)/lnp+1)}{p-1} \xrightarrow{n \rightarrow \infty} +\infty\]
which means the definition is well-defined. The upper bound of \(S_n\) holds here because integer \(n\) has at most \(\lfloor\frac{ln(n)}{lnp}\rfloor+1\) p-digits. When \(|x| = p^{-1/p-1}\), we notice that for \(n = p^k\), we have
\[|\frac{x^n}{n!}|_p  = p^{-p^k/p-1}\cdot p^{p^k-1/p-1} = p^{1/p-1} \]
Hence  the series diverges and the domain of the convergence is \(B_p(0,p^{-1/(p-1)})\). 
\newline

Some properties about the power series will be needed here for the following proof.
\begin{lemma}[analytic continuation] \label{cont}
    Let \(f(X)\) and \(g(X)\) be two formally power series over a complete non-archimedean field \(K\), and they all converge on the define domain \(D\) containing zero. If there exists a non-stationary convergent sequence \((a_n)_{n \in \n}\) of \(D\) such that \(f(a_n)=g(a_n)\), then \(f(X) = g(X)\).

    \begin{proof}
        The proof is similar to the classical proof. We define
        \[h(X) = f(X)-g(X) = \sum_{k\geq 1} c_k X^k\]
        with \(h(a_n) = 0\) for any \(n\). Assuming that \(h(X)\) is not zero, then we take \(r = \{\min{n \in \n| c_n \neq 0}\}\) the smallest non-zero index, then \(h(X)=X^rh_1(X)\), here \(h_1\) is defined by a power series with the non-zero constant cofficient, and it also converges on \(D\). Then by continuity, we have
        \[ \lim_{n \rightarrow \infty} h_1(a_n) = h_1(\lim_{n \rightarrow \infty}a_n) = h_1(0) = c_r \neq 0\]
        Hence for a large \(N\), \(h_1(a_N) \neq 0\). Moreover, non-stationary sequence \((a_n)_{n\in \n}\) implies \(a_N \neq 0\), so \(h(a_N) = a_N^rh_1(a_N)\neq 0\), absurd.
    \end{proof}

    \begin{lemma}[composition] \label{composition}
        Let \(f(X) = \sum_{n \geq 0} a_nX^n\) and \(g(X) = \sum_{m\geq 1}b_mX^m\) be two formal power series, let \(R\) be the radius convergence of \(f\). If \(x\) is an element of a complete non-archimedean field \(K\) which satisfies\\
        (1) \(g(x)\) converges. \\
        (2) \(|b_mx^m| < R\) for any \(m \geq 1\). \\
        then then the formal power series \(h(X) = f\circ g (X)\) converges at \(x\) with \(h(x)=f(g(x))\).

        \begin{proof}
            The proof can be founded in Cohen's book \cite[Chapter 4, proposition 4.2.7]{cohen2007number}.
        \end{proof}
    \end{lemma}
\end{lemma}

Logarithm and exponetial function keeps the same algebraic properties in p-adic context, here we just need several properties for applications to the solution of the equation.
\begin{proposition} \label{explog}
    Let \(a,b \in \cc_p\) with \(|a|_p, |b|_p < p^{-1/(p-1)} \), then\\
    (1) \(\exp (a+b)  = \exp (a) \exp(b)\) \\
    (2) \(|\log(1+a)|_p = |a|_p\)\\
    (3) \(\exp (\log (1+a)) = 1+a\)
    %(4) \(|\exp(a)|_p = 1\)
    \begin{proof}
        (1) \(|a+b|\leq max \{|a|,|b|\}<p^{-1/p-1}\), so \(\exp(a+b)\) exists. By a manipulation of power series
        \begin{align*}
           \exp(a)\exp(b) &= (\sum_{m=0}^{\infty}\frac{a^m}{m!})(\sum_{n=0}^{\infty}\frac{b^n}{n!}) \\
           &= \sum_{k\geq 0}\frac{1}{k!}\sum_{m+n=k}\frac{k!}{m!\cdot n!}a^mb^n\\
           &= \sum_{k\geq 0}\frac{1}{k!}(a+b)^k = \exp(a+b)
        \end{align*}
        we finish the proof.

        (2) Notice that \(v_p(n!)=v_p(n)+v_p((n-1)!)\) and \(v_p(n!)\geq 0\), which implies \(|n!|_p \leq |n|_p\). and we can estimate that
        \[v_p(\frac{a^{n-1}}{n!}) = (n-1)v_p(a)-v_p(n!) > \frac{n-1}{p-1} - \frac{n-S_n}{p-1} = \frac{S_n-1}{p-1} \geq 0\]
        Hence we can conclude that 
        \[|\frac{a^n}{n}|_p \leq |\frac{a^n}{n!}|_p = |\frac{a^{n-1}}{n!}|_p \cdot|a|_p < |a|_p\]        
        for any \(n \geq 2\). Therefore by the inequality of ultrametric, we can conlude the result, and notice here will still hold if \(|a|_{p} <1\).

        (3) Firstly we will check the condition of the lemma \ref{composition}. Let \(f(X) = \exp(X)\) and \(g(X) = \log(1+X)\), then \(|a| < p^{-1/p-1} <1\) impiles that \(g(a)\) converges. Notice that each term \((-1)^{m+1}\frac{x^m}{m}\) in \(g(a)\), we have estimated in the proof of (2), the absolute value is strictly less than the radius \(R = p^{-1/p-1}\), so by composition we proved that \(\exp(\log(1+a))\) converges. Let \(x_k = \frac{p^k}{p^k+1} <1\) be the sequence of \(\qq\), caculate its p-adic absolute value \(|x_k|_p=p^{-k} < R\) (to avoid the equality here, we convente \(k \geq 2\)), hence \(x_k\) is a non-stationary sequence converging to zero by p-adic absolute value. Finally by lemma \ref{cont}, we can conclude that \( \exp(\log(1+a)) = 1+a\) since formally power series \(\exp(log(1+X))\) has the same cofficient with \(1+X\).

        %(4) notice that
        %\[|\exp(a)-1| = |\sum_{n \geq 1}\frac{a^n}{n!}| < |a| <1 \]
        %Hence immediately by ultrametric
        %\[|\exp(a)| = |\exp(a)-1+1| = 1\]
    \end{proof}
\end{proposition}

\begin{remark}
    The method of proof (3) is to avoid discussing too much formal power series. Generally, we can prove the permanence of algebraic form 
    \[\exp(\log(1+X)) = 1+X\]
    without considering the convergence over a formal power series ring \(R[[X]]\) with \(R\) as a commutative \(\qq\)-algebra. The proof without analytic methods is not easy, it needs some combinatorial trick, a method via formal derivative can be founed in \cite[Chapter 3]{sambale2023invitation}.
\end{remark}


Applying (1) to (2), then we can get the identity
\[(1+a)^n = \exp(n \log(1+a)), \quad \forall n \in \n\]
For extending the definition for interpolation, i.e. let \((1+a)^x\) makes sense for any \(x \in \zp\), a traditional definition is based on the Newton's binomial theorem (see \cite[section 5.9]{gouvea1997p}), which needs some work and here we will not use binomial, so we consider the extension by p-adic  exponentials and logarithm, and notice that \(\n\) is dense in \(\z_p\), which makes the following extending be natural.

\begin{definition} \label{interpolate}
    Let \(a \in \cc_p\) with \(|a|_p < p^{-1/(p-1)}\), then the binomial interpolation can be defined by a p-adic analytic function
    \[f_a: \zp \to \cc_p, \quad x \mapsto \exp(x \log(1+a))\]     
    This construction satisfies \(f_a(n) = (1+a)^n\) for any integr \(n\).
\end{definition}

When fixing \(a\), we can estimate for any \(x \in \zp\)
\[|x\log(1+a)|_p = |x|_p|a|_p < p^{-1/p-1}\]
that means \(f_a\) is well-defined, and by convention we denote \(f_a(x) = (1+a)^x\).

Strassman's Theorem will be the crucial part in the proof, we give a version which is easy to use here:

\begin{theorem}[Strassman's Theorem] $ \\$
    Let \(f(X)\) be a non-zero element of the Tate algebra over \(\cc_p\), i.e. a formal power series with cofficient \((a_n)_{n \geq 0}\) converging to zero.
    \[f(X) = \sum_{n =0}^{\infty}a_n X^n = a_0 +a_1X +a_2X^2+...\]
    Let \(N = \max \{{m \in \n : |a_m|_p \geq |a_n|_p \text{ for all } n \in \n  }\)\}, then \(f: \zp \to \cc_p\) has at most \(N\) zeros.

    \begin{proof}
        It is rewritten from corollary 16.14.
    \end{proof}
\end{theorem}
    


