\section{The equation $x^3-2y^3=1$}
Now for solve the equation, we take \(K = \mathbb{Q}(\sqrt[3]{2})\) be the extension field of the rational numbers and we denote \(\theta = \sqrt[3]{2}\), then each element in \(K\) has the form
\[a+b\theta +c \theta^2 \quad  \text{with } a,b,c  \in \mathbb{Q}\]
Then we prove some properties of the field:
\begin{proposition} \label{qqsqrt}
    in \(\qqq\) we have

    - The unit group is isomorphic to \(\z/2\z \times \z\).

    - \(N_{\qqq / \qq}(a+b\theta+c\theta^2) = a^3+2b^3+4c^3-6abc\)

    - \(u = -1 + \theta \) is a fundamental unit.


    \begin{proof}
        Firstly we suppose that \(\sigma: \qqq \hookrightarrow  \cc \) is a field embedding, then surely \(\sigma(1)=1\). Let \(f(X) = X^3-2\) be a polynomial, and notice that \(f(\theta) = 0\), then
        \[0 = \sigma(f(\theta)) = f(\sigma(\theta))\]
        Clearly \(\sigma(\theta)\) must be the root of \(f\) in \(\cc\), so we can conclude the roots are \(\theta, \theta w , \theta w^2\), where \(w = e^{2i\pi /3}\). Hence the unique real embedding is \(\sigma=id\) and there are two conjugate complex embedding, which means \(r=1\) and \(s=1\). For the group of roots of unity, we notice that \(\qqq \subset \rr\) as a subfield, and \(x^n=1\) only has possible solutions \(\{\pm 1\}\) in \(\rr\) for any \(n \in \mathbb{N}\), so \(\mu_K = \{\pm 1\} \cong \z/2\z\).

        For the norm we consider the \(\qq\)-linear map \(l_x\) with \(x=a+b \theta + c \theta^2\), then
        \[l_x(1) = a+b \theta + c \theta^2, l_x(\theta )= 2c+a \theta + b \theta ^2, l_x(\theta ^2) = 2b+ 2c \theta + a \theta ^2 \]
        so we can conclude the norm by 
        \[det{[l_x]_{\{1,\theta,\theta ^2\}}} = \begin{vmatrix}
            a & 2c & 2b \\
            b & a & 2c \\
            c & b & a
        \end{vmatrix} = a^3+2b^3+4c^3-6abc\]
        and we take \(u= -1 + \theta\), then \(N(u) = -1+2=1\), so it is a unit.

        Finally we prove that \(u\) is exactly a fundamental unit by contradiction. Assuming that \(\eta>1\) is a fundamental unit, and notice that \(0<u<1\), so there exists a integer \(k \geq 1\) such that \(u = \eta^{-k}\). In this case we have negative discriminant \(\Delta = -108\), then by lemma \ref{criterion} we can estimate \(\eta > \sqrt[3]{21}\), then 
        \[-1 + \sqrt[3]{2} = \eta^{-k} < (\sqrt[3]{21})^{-k}\]
        It only holds for \(k=1\), which means \(u\) is the largest positive unit less than one, so \(u\) can be choosen as a fundamental unit.
        
    \end{proof}
\end{proposition}

Return to the original equation, now we can give a equivalent statement:
\begin{proposition} \label{equi}
    The integral solutions of the equation \(x^3-2y^3=1\) are
    \[\{(x,y)\in \z| x-y\theta = u^k, \text{for some } k \in \z\}\]

    \begin{proof}
        We notice that \(x^3-2y^3 = 1 \) can be rewritten as \(N_{\qqq / \qq}(x-y\theta) = 1\), which means that \(x-y\theta\) is a unit. And by the Dirichlet's unit theorem, the unit group of \(\mathcal{O}_K\) is of the form \(\pm u^k\). Notice that \(N_{\qqq / \qq}(-1) = -1\), so
        \[N_{\qqq / \qq}(-u^n) = N_{\qqq / \qq}(-1)(N_{\qqq / \qq}(u))^n = -1, \quad \forall n \in \z \]
        Hence the integral solution is of the form \(u^k\) in \(\qqq\).
    \end{proof}
\end{proposition}

Notice that if \(k=0\) we get the trival solution \((1,0)\); if \(k=1\), we find another solution \((-1,-1)\); By known result, they are exact the only two solutions, so we need to prove that for any other \(k\), \(x-y\theta = u^k\) has no solution, we will prove that for any other \(k\), the coefficient of \(u^k\) with respect to base vector \(\theta^2\) is non-zero. For the case \(k<0\), we notice that \(u^{-1} = 1+\theta +\theta^2\) and use multinomial formula
\[(1+\theta+ \theta^2)^k = \sum_{i+j+k=n} \frac{n!}{i!j!k!}\theta ^{j+2k}\]
with \(\theta ^3= 2\) we can rewrite it to get a linear combination of \(\{1,\theta,\theta ^2\}\), clearly here the coefficient of \(\theta ^2\) will not be zero so the choice of \(k\) will be limited to be more than zero. However, when \(k \geq  2\) we will find that it is difficult to analyse, for example
\begin{align*}
    u^2 &= 1- 2\theta +\theta^2 \\
    u^3 &= 1+3\theta-3\theta^2 \\
    u^4 &= -7-2\theta+6\theta^2\\
    &...
\end{align*}
 
The problem here is that it is diffcult to formulate \(u^{k}\) since there exists negative cofficient in \(-1+\theta\), it is not easy to deduce that whether the cofficient of \(\theta^k\) will vanish in a certain \(k\) or not, the argumment here will be not clear, so we will turn towards to the p-adic method.
\newline 

Firstly, we notice that \(\sqrt[3]{2} \notin \qq_3\) since by inspection \(x^3 = 2 \mod 9\) has no solution, so we still consider the finite extension by adjoining \(\theta = \sqrt[3]{2}\) to construct, then we get a new field \(\qq_3(\theta) \cong \qq_3[X]/(X^3-2)\) containing \(\qq_3\) as a subfield, with any element of the form
\[a+b\theta +c \theta^2 \quad  \text{with } a,b,c  \in \mathbb{Q}_3\]
The norm can be simliarly caculated like in \(\qq(\sqrt[3]{2})\) (proposition \ref{qqsqrt}), so we can uniquely extend the absolute value of \(\qq_3\) as following:
    \[|a+b\theta+c\theta^2| = \sqrt[3]{|a^3+2b^3+4c^3-6abc|_3}\]


    Notcie that \(\qq_3(\theta)\) is a subfield of \(\cc_3\), so we will consider the binomial interpolation in this field, observing that \(|u-1| =1 > 3^{-1/2}\) prevents us from directly using interpolation, and notice that \(|u^3-1| = 3^{-1} < 3^{-1/2}\), so we will interpolate on \(u^3\).

    \begin{lemma}
        There exists a convegrent power series \(h(X)\) with coefficient in \(\z_3[\theta]\) such that
        \begin{equation} \label{eq1}
            (u^3)^x = 1 + (3\theta-3\theta^2)x+9xh(x)
         \end{equation}
        for any \(x \in \z_3\).

        \begin{proof}
            By definition \ref{interpolate}, interpolation \((u^3)^x = \exp(x \log(u^3))\) is well-defined here, so we have
            \[(u^3)^x = 1+ \log(u^3)x + 9x^2 \sum_{k \geq 0} \frac{(\log u^3)^{k+2}}{9(k+2)!}x^k\]
            we can estimate the valuation by proposition \ref{explog} property (2)
            \begin{align*}
                v(\frac{(\log u^3)^{k+2}}{9(k+2)!}) &= (k+2)-2-\frac{(k+2)-S_{k+2}}{2} \\
                &= \frac{k+S_{k+2}-2}{2} \geq 0
            \end{align*}
            so there exists a convegrent power series \(h'(X)\) with coefficient in \(\z_3[\theta]\) such that 
            \begin{equation} \label{l1}
                (u^3)^x = 1+ \log(u^3)x + 9x^2h'(x)
            \end{equation}
            And we rewrite \(\log(u^3)\) by the definition of p-adic logarithm
            \[\log(u^3) = u^3-1 + 9\sum_{k \geq 2} (-1)^{k+1}\frac{(u^3-1)^k}{9k}\]
            similarly we estimate the valuation for any \(k\geq 2\)
            \begin{align*}
                v((-1)^{k+1}\frac{(u^3-1)^k}{9k}) &= k-2-v_3(k) \\
                \geq k-2-\frac{\ln(k)}{\ln(3)} \geq 0
            \end{align*}
            so there exists a element \(a \in \z_3[\theta]\) such that
            \begin{equation} \label{l2}
                \log(u^3) = u^3-1 + 9a
            \end{equation}
            Plugging (\ref{l2}) to (\ref{l1}), then we get
            \[(u^3)^x = 1+(u^3-1)x+9x[a+xh'(x)]\]
            here \(h(X) = a+Xh'(X)\) is the power series we hope.
        \end{proof}


    \end{lemma}

    \begin{theorem} \label{x3-2y3=1}
        The only solutions to the integral equation on
\[
x^{3}-2y^{3}=1
\]
are\/ $(x,y)=(1,0)$ and\/ $(x,y)=(-1,-1)$.

    \begin{proof}
        By the proposition \ref{equi}, it is sufficient to study the cofficient with respect to \(\theta^2\) to show that the integer power of the fundamental unit \(u^n\) can not be of the form \(x-y \theta\) unless \(n=0,1\). We interpolate \(u^n\) here by defining the functions \(f_r(x) = u^{r}\cdot (u^3)^x\) with \(r=0,1,2\), it is reasonable here since
        \[f_0(\z)\cup f_1(\z) \cup f_2(\z) = u^{\z}\]
        and for any \(r=0,1,2\) we can write the \(f_r\) as the form of linear combination as following
         \[f_r(x) = (\sum_{k \geq 0}a_kx^k)+(\sum_{k \geq 0}b_kx^k)\theta + (\sum_{k \geq 0}c_kx^k)\theta^2\]

         -When \(r=0\), by equation (\ref{eq1}) we have
         \[f_0(x) = 1+ 3x\cdot \theta + (-3\theta^2x+9xh(x))\]
         In detail, by writing \(h(x)\) as the form of linear combination
         \[h(x) = h_1(x)+h_2(x)\cdot\theta + h_3 \cdot \theta^2\]
         with \(h_1,h_2,h_3\) the convergent power series defined on \(\z_3\), so again
         \[f_0(x) = (1+9xh_1(x)) + (3x+9xh_2(x))\cdot \theta + (-3x+9xh_3(x)) \cdot \theta^2\]
         we apply Strassman's theorem to \(-3x+9xh_3(x) = 0\), and notice that the coefficient of \(x\) has valuation 1 while the other have that at least 2, hence we can conclude that \(N=1\) and \(x=0\) is the unique solution, which corresponds to \(n=0\).
         
         -When \(r=1\), simliarly the equation can be rewritten as
         \[f_1(x) = [-1-6x-9xh_3(x)+18xh_1(x)]+[1-3x-9xh_2(x)+9xh_3(x)]\cdot \theta + [6x+9xh_2(x)-9xh_1(x)]\cdot \theta^2\]
         applying Strassman's theorem to \(6x+9x(h_1+h_2)(x) = 0\), we can conclude that \(N=1\) and \(x=0\) is the unique solution, whci corresponds to \(n=1\)

         - When \(r=2\), simliarly the coefficient with respect to \(\theta^2\) is \[1-9x+9(h_3(x)-2h_2(x)+h_3(x)) \]
         notice that the constant coefficient \(|1| = 1\), which strictly greater than any other coefficient, hence this expression does not vanish on \(\z_3\) by Strassman's theorem.

         In conclusion, we can conclude the solution of the integral equation \(x^3-2y^3=1\) by proposition \ref{equi}, when \(n \equiv 0 \mod 3\),  the only solution is \((1,0)\) which corresponds to \(r=0, x=1\); when \(n \equiv 1 \mod 3\), the only solution is \((-1,-1)\) which corresponds to \(r=1,x=0\); when \(n \equiv 2 \mod 3\), no solution will exists.
    \end{proof}
    \end{theorem}


   




