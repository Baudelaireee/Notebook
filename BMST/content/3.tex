\section{Interpolation Method}

Now we will solve the equation by using p-adic interpolation method. Firstly, we notice that \(\sqrt[3]{2} \notin \qq_3\) by locally observing that \(x^3 = 2 \mod 9\) has no solution, so we still consider the finite extension by adjoining \(\theta = \sqrt[3]{2}\) to construct, then we have the similar result.

\begin{proposition}
    In \(\qq_3 (\theta)\) we have

    - This field is a complete non-archimdean field with the absolute value:
    \[|a+b\theta+c\theta^2| = \sqrt[3]{|a^3+2b^3+4c^3-6abc|_3}\]
\end{proposition}

    In this field, observing that \(|u-1| =1 > 3^{-1/2}\) prevents us from directly using interpolation, and notice that \(|u^3-1| = 3^{-1} < 3^{-1/2}\), hence we will interpolate on \(u^3\).

    \begin{theorem}
        The only solutions to the integral equation on
\[
x^{3}-2y^{3}=1
\]
are\/ $(x,y)=(1,0)$ and\/ $(x,y)=(-1,-1)$.

    \begin{proof}
         Let \(f: \z_3 \to \qq_3(\theta)\) be the p-adic analytic function defined by \(f(x) = \exp({x\log{u^3}})\), and \(f|_{\z}(n) = u^{3n}\), it is well-efined by \textbf{lemma...}. In particular
         \[\log{u^3} \equiv  3\theta-3\theta^2 \mod{9 \z_3}\]
         hence 
         \begin{equation} \label{eq1}
            \exp(x \log u^3) = 1 + (3\theta-3\theta^2)x+9xh(x)
         \end{equation}
         for some convegrent power series \(h(X)\) with cofficient on \(\z_3(\theta)\). Since \(\qq_3(\theta)\) is a vector space under basis \(\{1, \theta, \theta^2\}\), then \(f(x)\) can be denoted by three power series with respect to basis as following
         \[f(x) = (\sum_{k \geq 0}a_kx^k)+(\sum_{k \geq 0}b_kx^k)\theta + (\sum_{k \geq 0}c_kx^k)\theta^2\]
         and we will study the cofficient with respect to \(\theta^2\) to show that \(u^n\) can not be of the form \(x-y \theta\) unless \(n=0,1\). we take \(f_r(x) = u^{r}f(x)\) with \(r=0,1,2\).

         -When \(r=0\), the equation (\ref{eq1}) can be rewritten as
         \[f_0(x) = 1+ 3x\cdot \theta + (-3\theta^2x+9xh(x))\]
         In detail, by writing \(h(x)\) as the form of linear combination
         \[h(x) = h_1(x)+h_2(x)\cdot\theta + h_3 \cdot \theta^2\]
         with \(h_1,h_2,h_3\) the convegrent power series defined on \(\z_3\), so again
         \[f_0(x) = (1+9xh_1(x)) + (3x+9xh_2(x))\cdot \theta + (-3x+9xh_3(x)) \cdot \theta^2\]
         we apply Strassman's theorem to \(-3x+9xh_1(x) = 0\), and notice that the cofficient of \(x\) is \(a_1 \equiv 3 \mod 9\z_p\) and the other cofficients are \(a_i \equiv 0 \mod 9\z_p \), hence we can conclude that \(N=1\) and \(x=0\) is the unique solution.
         
         -When \(r=1\), simliarly the equation can be rewritten as
         \[f_1(x) = [-1-6x-9xh_3(x)+18xh_1(x)]+[1-3x-9xh_2(x)+9xh_3(x)]\cdot \theta + [6x+9xh_2(x)-9xh_1(x)]\cdot \theta^2\]
         applying Strassman's theorem to \(6x+9x(h_1+h_2)(x) = 0\), we can conclude that \(N=1\) and \(x=0\) is the unique solution.

         - When \(r=2\), simliarly the cofficient with respect to \(\theta^2\) is \[1-9x+9(h_3(x)-2h_2(x)+h_3(x)) \]
         notice that the constant cofficient \(|1| = 1\), which strictly greater than any other cofficient, hence no solution for \(x\) such that the cofficient turns zero by Strassman's theorem.

         In conclusion, we can conclude the solution of the integral equation \(x^3-2y^3=1\) by proposition \ref{equi}, when \(n \equiv 0 \mod 3\),  the only solution is \((1,0)\) which corresponds to \(r=0, x=1\); when \(n \equiv 1 \mod 3\), the only solution is \((-1,-1)\) which corresponds to \(r=1,x=0\); when \(n \equiv 2 \mod 3\), no solution will exists.
    \end{proof}

    
    \end{theorem}


