\section{Interpolation Method}

Now we will solve the equation by using p-adic interpolation method. Firstly, we notice that \(\sqrt[3]{2} \notin \qq_3\) by locally observing that \(x^3 = 2 \mod 9\) has no solution, so we still consider the finite extension by adjoining \(\theta = \sqrt[3]{2}\) to construct, then we have the similar result.

\begin{proposition}
    In \(\qq_3 (\theta)\) we have

    - This field is a complete non-archimdean field with the absolute value:
    \[|a+b\theta+c\theta^2| = \sqrt[3]{|a^3+2b^3+4c^3-6abc|_3}\]
\end{proposition}

    In this field, observing that \(|u-1| =1 > 3^{-1/2}\) prevents us from directly using interpolation, and notice that \(|u^3-1| = 3^{-1} < 3^{-1/2}\), hence we will interpolate on \(u^3\).

    \begin{theorem} \label{x3-2y3=1}
        The only solutions to the integral equation on
\[
x^{3}-2y^{3}=1
\]
are\/ $(x,y)=(1,0)$ and\/ $(x,y)=(-1,-1)$.

    \begin{proof}
         Let \(f: \z_3 \to \qq_3(\theta)\) be the p-adic analytic function defined by \(f(x) = \exp({x\log{u^3}})\), and \(f|_{\z}(n) = u^{3n}\), it is well-efined by \textbf{lemma...}. In particular
         \[\log{u^3} \equiv  3\theta-3\theta^2 \mod{9 \z_3}\]
         hence 
         \begin{equation} \label{eq1}
            \exp(x \log u^3) = 1 + (3\theta-3\theta^2)x+9xh(x)
         \end{equation}
         for some convegrent power series \(h(X)\) with cofficient on \(\z_3(\theta)\). Since \(\qq_3(\theta)\) is a vector space under basis \(\{1, \theta, \theta^2\}\), then \(f(x)\) can be denoted by three power series with respect to basis as following
         \[f(x) = (\sum_{k \geq 0}a_kx^k)+(\sum_{k \geq 0}b_kx^k)\theta + (\sum_{k \geq 0}c_kx^k)\theta^2\]
         and we will study the cofficient with respect to \(\theta^2\) to show that \(u^n\) can not be of the form \(x-y \theta\) unless \(n=0,1\). we take \(f_r(x) = u^{r}f(x)\) with \(r=0,1,2\).

         -When \(r=0\), the equation (\ref{eq1}) can be rewritten as
         \[f_0(x) = 1+ 3x\cdot \theta + (-3\theta^2x+9xh(x))\]
         In detail, by writing \(h(x)\) as the form of linear combination
         \[h(x) = h_1(x)+h_2(x)\cdot\theta + h_3 \cdot \theta^2\]
         with \(h_1,h_2,h_3\) the convegrent power series defined on \(\z_3\), so again
         \[f_0(x) = (1+9xh_1(x)) + (3x+9xh_2(x))\cdot \theta + (-3x+9xh_3(x)) \cdot \theta^2\]
         we apply Strassman's theorem to \(-3x+9xh_1(x) = 0\), and notice that the cofficient of \(x\) is \(a_1 \equiv 3 \mod 9\z_p\) and the other cofficients are \(a_i \equiv 0 \mod 9\z_p \), hence we can conclude that \(N=1\) and \(x=0\) is the unique solution.
         
         -When \(r=1\), simliarly the equation can be rewritten as
         \[f_1(x) = [-1-6x-9xh_3(x)+18xh_1(x)]+[1-3x-9xh_2(x)+9xh_3(x)]\cdot \theta + [6x+9xh_2(x)-9xh_1(x)]\cdot \theta^2\]
         applying Strassman's theorem to \(6x+9x(h_1+h_2)(x) = 0\), we can conclude that \(N=1\) and \(x=0\) is the unique solution.

         - When \(r=2\), simliarly the cofficient with respect to \(\theta^2\) is \[1-9x+9(h_3(x)-2h_2(x)+h_3(x)) \]
         notice that the constant cofficient \(|1| = 1\), which strictly greater than any other cofficient, hence no solution for \(x\) such that the cofficient turns zero by Strassman's theorem.

         In conclusion, we can conclude the solution of the integral equation \(x^3-2y^3=1\) by proposition \ref{equi}, when \(n \equiv 0 \mod 3\),  the only solution is \((1,0)\) which corresponds to \(r=0, x=1\); when \(n \equiv 1 \mod 3\), the only solution is \((-1,-1)\) which corresponds to \(r=1,x=0\); when \(n \equiv 2 \mod 3\), no solution will exists.
    \end{proof}
    \end{theorem}


    similar technic we can apply to completly solve the diophantine equation of the form
    \begin{equation} \label{eq2}
        x^3+dy^3=1
    \end{equation}
    which we call it Skolem's equation. Skolem is influenced by the work of Thue in the beginning of the 19th. Thue improved the Liouville's approximation theorem to give a lower approxiamtion exponent \(\tau(d) = \frac{d}{2}+1+\epsilon\), which shows that the number of the integral solution of equation (\ref{eq2}) will be finite (see \cite[Chapter 11]{silverman2009arithmetic}). However, this method of diophantine approxiamtion is not effective, in 1937 Skolem make use of p-adic interpolate method to give a same answer that the solution of the equation (More generally, he states for a irreducible homogeneous polynoimal) will be finite, even more precisely, at most two solution.

    
    \begin{theorem}[Skolem] There exists at most one non-trival solution for the Integral equation
        \[x^3+dy^3=1\]
    where \(d \in \z\).
        \begin{proof}
            If \(d\) is a perfect cubic, then the solution will be related to the equation \(x^3+y^3=1\), which only has two solution \((1,0)\) and \((0,1)\), so there exists at most one non-trival solution. If \(d\) is not perfect cubic, we consider the field extension \(K = \qq(\theta)\) with \(\theta = \sqrt[3]{d}\). By unit theorem, we denote \(u\) is the positive unit, and then if \((x,y)\) is a Integral solution, \(x+y\theta\) will be of the form \(u^k\) form some integer \(k\).

            Suppose that we have two non-trival solution \((x_1,y_1)\) and \((x_2,y_2)\), here \(x_iy_i \neq 0\) and then there exists non-zero integer \(p_1\) and \(p_2\) such that \(x_1+y_1\theta = u^{p_1}\) and \(x_2+y_2\theta = u^{p_2}\). Let \(p_1/p_2 = n_1/n_2\) with \(\gcd(n_1,n_2)=1\), then \(n_1/n_2\) or \(n_2/n_1\) can be seen as a p-adic integer. It is sufficient to assume that \(N = n_2/n_1 \in \z_3\), then 
            \[x_2+y_2\theta = u^{p_2} = u^{Np_1} = (x_1+y_1\theta)^N\]
            Notice that \[(x_1+y_1\theta)^3 = 1+3xy(x\theta+y\theta^2)\]
            we put \(N= 3M+r\) with \(M \in \z_3\) and \(r=0,1,2\), then we have 
            \[x_2+y_2\theta = [1+3xy(x\theta+y\theta^2)]^M(x+y\theta)^r\]
            with \(x=x_1\) and \(y=y_1\). We consider it in the completion of the finite extension \(\qq (\theta)\) by 
            \[L \cong \qq_3 \otimes_{\qq} \qq(\theta)\]
            then there exists a convegrent series \(B \in \z_3[\theta]\) such that
            \begin{equation} \label{eq3}
                x_2 + y_2 \theta = \left( 1 + 3M x y (x\theta + y\theta^2) + 9M x^2 y^2 B \right) (x + y\theta)^r 
            \end{equation}
            write \(B=B_0+B_1\theta+B_2\theta^2\) with \(B_1,B_2,B_3 \in \z_3\), and then rewrite equation (\ref{eq3}) as the linear combination of \(\{1,\theta,\theta\}\), the cofficient with respect to \(\theta^2\) must be zero, so we have 
            \[
\begin{cases}
3M x y^2 (1 + 3x B_2)=0 & \text{for } r = 0, \\
3M x^2 y^2 \left( 2 + 3(y B_1 + x B_2) \right)=0 & \text{for } r = 1, \\
y^2 \left( 1 + 9M x^2 (x + B_2 x^2 + 2B_1 x y + B_0 y^2) \right)=0 & \text{for } r = 2.
\end{cases}
\]
            Notice that notice that \(N \neq 0, 1\), which means for \(r=0\) or \(r=1\) we must have \(M\neq 0\), then we can divide \(3Mxy^2, 3Mx^2y^2, y^2\) respectively, and then we can get contradiction by modulo 3 (\(1 \equiv 0, 2 \equiv 0, 1\equiv 0\) respectively).
                
        \end{proof}
    \end{theorem}

    This result can be further refined, and we can provide a necessary and sufficient condition for the existence of nontrivial solutions to the Skolem equation. Review the proof of theorem \ref{x3-2y3=1}, the non-trival 
    solution is just the fundamental unit. Notice that in our case, we have 4 choice for fundamental unit: \(u,-u,1/u,-1/u\), and there are exactly two positive unit \(u,1/u\), so the existence depends on the detail of the unit.





