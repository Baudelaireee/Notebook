\section{Other method(not formally)}
Now we consider the other possible method corresponding to the integral solution. We founded that if \(2\) is a perfect cubic number, then the solution will be very easy, but unfortunately we can not do like that. However, p-adic number system gives us a method to extend the field, surely we consider a prime \(p\) (for example,p =5?) such that \(\sqrt[3]{2} \in \z_p\), then we just need to study the the p-adic integral equation
\[x^3+y^3=1\]
with \(x,y \in \z_p\).

For example we take the set \(S\) as the solution of the equation,
then for any \(x,y \in S\), there exists sequences \((x_n)_{n \in \n}\) and \((y_n)_{n \in \n}\) with
\(x_n, y_n \in \z_p/p^n\z_p\) such that
\[\begin{cases}
    x_n \equiv x_{n+1} \mod {p^n} \\
    y_n \equiv y_{n+1} \mod {p^n} \\
    x_n^3+y_n^3 \equiv 1 \mod {p^n}
\end{cases}\]

Fix \(n\), and we take \((x_n,y_n)\) to form a sequence \(S_n\), the the space of the solution can be derived by the inverse limit as following
\[S = \varprojlim_{n} S_n\]

By computing the case \(p=3\), we take \(A_n = \{(x_n,y_n)| x^3_n+y^3_n \equiv 1 \mod 3^n \}\), some example as following
\begin{align*}
    A_1 &= \{\, (1,0),\ (0,1),\ (2,2) \,\} \\[1ex]
    A_2 &= \{\, 
        (0,1),\ (3,4),\ (6,7),\ 
        (1,0),\ (4,3),\ (7,6)
    \,\} \\[1ex]
    A_3 &= \{\, 
    \begin{array}[t]{llll}
        (0,1), & (9,10), & (18,19), & \\
        (3,4), & (12,13), & (21,22), & \\
        (6,7), & (15,16), & (24,25), & \\
        (1,0), & (10,9), & (19,18), & \\
        (4,3), & (13,12), & (22,21), & \\
        (7,6), & (16,15), & (25,24) &
    \end{array}
    \,\}
    \end{align*}

    With that we can do selecting: notice that the possible original solution for lifting are just three possible, so there will exist three path to consider, so we can do lifting as following

    -starting from \((1,0)\):
    \[(1,0) \rightarrow \begin{matrix}
        (1,0) \\
        (4,3) \\
        (7,6)
    \end{matrix} \rightarrow
    \begin{matrix}
        (1,0) & (10,9) & (19,18) \\
        (4,3) & (13,12) & (22,21) \\
        (7,6) & (16,15) & (25,24)
    \end{matrix}\]

    starting from \((0,1)\) is symetrical as above, but pay attention that there exists no lifting when starting from \((2,2)\). Then we should consider all possible lifting, that is motivated from Hensel's lemma, for example \((1,0)\) is exactly a solution for the equation \(x^3+y^3=1\), and we can find the a lifting
    \[(1,0) \rightarrow (1,0) \rightarrow (1,0) \rightarrow ...\]
    The choice of the prime \(p=3\) is really terrible here, since for \(f(x,y) = x^3+y^3-1\), the partial derivative \(f_x \equiv f_x \equiv 0 \mod 3\), that means the algebraic curve we consider is not smooth? so we may consider the other prime.

    \textbf{Question:}
    As what u show, and i have verified that there are usually infinite solutions for a p-adic integral equation, it can be done by Hensel's lemma (fix some certain \(y\) and then use Hensel's lemma for one-variable polynomial to do lifting). However, in this case eahc solution \((x,y) \in \z_3^2\) will satisfying a properties, it must be lifted from \((1,0)\) or \((0,1)\), that means there will be two paths (each paths maybe contain infinite 3-adic solution), so the solution can be described as following
    \[\text{solution lifted from} (1,0) + \text{solution lifted from} (0,1) = \text{all 3-adic solutions
    }\]
    
    I think it is not easy to precise each compoent of solutions to get the exact integer solution instead p-adic integer solutions.

    We consider above process from the view of scheme (i am not very fimilar to that).
    we consider a algebraic variety by letting \(f = (X^3+Y^3-1)\)
    \[X = \text{Spec}(\z_p[X,Y]/f)\]
    so all solution in \(\z_p\) can be denoted by \(X(\z_p)\), consider the inverse limit
    \[X(\z_p) = \varprojlim_{n} X_n(\z/p^n\z)\]
    where \(X_n = \text{Spec}((\z/p^n\z)(X,Y)/f)\) the affine scheme defined in a finite ring. In particular, when \(n=1\), \(X_n\) defines a algebraic curve in \(\ff_p\). So our original question is to find \(X(\z_p) \cap \z^2\).
