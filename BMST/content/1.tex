\section{Dirichlet's Unit theorem}


\begin{theorem}[Dirichlet's unit theorem] $\\$
    Let \(K\) be a algebraic number field with \(r\) real embeddings and \(2s\) complex embeddings, and let \(\mathcal{O}_K\) be its integer ring, then its unit group has the following structure:
    \[\mathcal{O}_K^\times \cong \mu(K) \times \mathbb{Z}^{r+s-1}\]
    where \(\mu(K)\) is the group of roots of unity in \(K\), and it is a finite cyclic gourp.

    \begin{proof}
        A standard proof can be founded in Neukirch's book \cite[section 1.7]{neukirch2013algebraic}, here we will just consider the case of \(r=1\) and \(s=1\), i.e. a extension \([K:\qq] =3\). Suppose that \(\sigma_r\) and \(\sigma_s\) are the real embedding and one of the complex embedding, then a unit \(u \in \mathcal{O}_K^\times \) satisfying \(|\sigma_r(u)||\sigma_s(u)|^2 = 1\). Hence we consider a hyperplan of \(\rr^2\)
        \[H:= \{(a,b)\in \rr^2| a+b=0\}\]
        then we will naturally get a exact sequence \[\begin{tikzcd}
	1 && {\mu_K} && {\mathcal{O}_K^\times} && l(H) && 0
	\arrow[from=1-1, to=1-3]
	\arrow["e", from=1-3, to=1-5]
	\arrow["l", from=1-5, to=1-7]
	\arrow[from=1-7, to=1-9]
\end{tikzcd}\]
         here \(e\) is a trival embedding by \(e(a) = a\), \(l\) is the logarithm map (in the real sense) defiend by
         \[u\mapsto (\log|\sigma_r(u)|, \log|\sigma_s(u)|)\]
         which is a homomorphism from the multiplicative group to the additive group, with \(\ker l = \{u\in {\mathcal{O}_K^\times}||\sigma_r(u)|=|\sigma_s(u)|=1\} = \mu_K\), it holds generally by Kronecker's theorme. so immediately we have the isomorphic
         \[{\mathcal{O}_K^\times}/\mu_K \cong l(H)\]
         Then we need to prove that \(l(H)\) is a nontrivial discrete subgroup of \(H\), i.e. a complete lattice of \(H\), which ensures \(l(H) \cong \z\). We consider the embedding \(j: {\mathcal{O}_K^\times} \to \rr \times \cc\) by 
         \[u \mapsto (\sigma_r(u),\sigma_s(u))\]
         Notice that integr ring \({\mathcal{O}_K^\times}\) is a free \(\z\)-module, then there exists integral base \(\{w_1,w_2,w_3\}\) such that for any \(u \in {\mathcal{O}_K^\times}\), there exists \(x,y,z \in \z\) such that 
         \[u = xw_1+yw_2+zw_3\] hence it invites a integral base for \(j({\mathcal{O}_K^\times})\) by
         \begin{align*}
            j(u) &= x \begin{pmatrix}
                w_1 \\
                \sigma_s(w_1)
             \end{pmatrix}+y\begin{pmatrix}
                w_2 \\
                \sigma_s(w_2)
             \end{pmatrix}+z\begin{pmatrix}
                w_3 \\
                \sigma_s(w_3)
             \end{pmatrix} \\
             &= xe_1+ye_2+ze_3
         \end{align*}
        hence under the base \(B=\{e_1,e_2,e_3\}\) we can see \(j({\mathcal{O}_K^\times})\) as the integer lattice of \(\rr \times \cc\). Now for any \((\log|\sigma_r(u)|,\log|\sigma_s(u)|) \in l(H)\), we take a voisinage \(V\) of the point, then \(\overline{j\circ l^{-1}(V)}\) implies a compact set of \(\rr \times \cc\), so it must contain finite integer lattice under the base \(B\), therefore \(V\) covers finite points, so \(l(H)\) is discrete.

        Finally \(l(H)\) must be nontrivial, it is not clear and even difficult, it is essential to prove the existence of the nontrival unit of \(\mathcal{O}_K^{\times}\)...
    \end{proof}
\end{theorem}

Although unit theorem can show us the structure of the unit, but it is diffcult to give a perfect algorithm to how to exactly compute the fundamental unit, here it is a criterion about the fundamental unit.
\begin{lemma} \label{criterion}
    Let \(K\) be a cubic extension of \(\qq\) with negative discriminant, and let \(u\) be the fundamental unit with \(u>1\), then 
    \[|\Delta_K| < 4 u^3 + 24\]

    \begin{proof}
        
    \end{proof}
\end{lemma}

A more strong estimation about the upper bound of the fundamental unit in a cubic field can be founded in Box's thesis \cite[Theorem 1.82]{box2014introduction}, that shows for a cubic field \(K = \qq(\sqrt[3]{a})\) with \(d = |\Delta_K|\), a unit \(u>1\) is a fundamental unit if and only if 
\[u < (\frac{d-32+\sqrt{d^2-64d+960}}{8})^{2/3}\]
