\section{Pre}


\begin{theorem}[Dirichlet's unit theorem] $\\$
    Let \(K\) be a number field with \(r\) real embeddings and \(s\) pairs complex embeddings, and let \(\mathcal{O}_K\) be its integer ring, then its unit group has isomorphic structure:
    \[\mathcal{O}_K^\times \cong \mu(K) \times \mathbb{Z}^{r+s-1}\]
    where \(\mu(K)\) is the group of roots of unity in \(K\), and it is a finite cyclic gourp.
\end{theorem}

Take \(K = \mathbb{Q}(\sqrt[3]{2})\) be the extension field of the raitional number,  and we denote \(\theta = \sqrt[3]{2}\), then each element in its has the form
\[a+b\theta +c \theta^2 \quad  \text{with } a,b,c  \in \mathbb{Q}\]
Then we prove some properties of the field:
\begin{proposition}
    in \(\qqq\) we have

    - \(r=1\) and \(s=1\).

    - \(N_{\qqq / \qq}(a+b\theta+c\theta^2) = a^3+2b^3+4c^3-6abc\)

    - \(u = 1 + \theta + \theta^2\) is a unit and its inverse is \(v= -1+\theta\).

    - The group of the unity is \(\mu = \{\pm 1\}\)


    \begin{proof}
        Firstly we suppose that \(\sigma: \qqq \hookrightarrow  \cc \) is a field embedding, then surely \(\sigma(1)=1\). Let \(f(X) = X^3-2\) be a polynoimal, and notice that \(f(\theta) = 0\), then
        \[0 = \sigma(f(\theta)) = f(\sigma(\theta))\]
        Clearly \(\sigma(\theta)\) must be the root of \(f\) in \(\cc\), so we can conclude the roots are \(\theta, \theta w , \theta w^2\), where \(w = e^{2i\pi /3}\). Hence the unique real embedding is \(\sigma=id\) and there are two conjugate complex embedding.

        For the norm we conisder the \(\qq\)-linear map \(l_x\) with \(x=a+b \theta + c \theta^2\), then
        \[l_x(1) = a+b \theta + c \theta^2, l_x(\theta )= 2c+a \theta + b \theta ^2, l_x(\theta ^2) = 2b+ 2c \theta + a \theta ^2 \]
        so we can conclude the norm by 
        \[det{[l_x]_{\{1,\theta,\theta ^2\}}} = \begin{vmatrix}
            a & 2c & 2b \\
            b & a & 2c \\
            c & b & a
        \end{vmatrix} = a^3+2b^3+4c^3-6abc\]
        and we take \(u= 1 + \theta + \theta^2\), then \(N(u) = 1+2+4-6=1\), so it is a unit.

        For the group of the unity, we notice that \(\qqq \subset \rr\) as a subfield, and \(x^n=1\) only has possible solutions \(\{\pm 1\}\) in \(\rr\) for any \(n \in \mathbb{N}\), so we can conclude our result.
    \end{proof}
\end{proposition}

Return to the original equation, now we can give a equivalent statement:
\begin{proposition}
    The integral solution of the equation \(x^3-2y^3=1\) is
    \[\{(x,y)\in \z| x-y\theta = u^k, \text{for some } k \in \z\}\]

    \begin{proof}
        We notice that \(x^3-2y^3 = 1 \) can be rewritten as \(N_{\qqq / \qq}(x-y\theta) = 1\). And by the Dirichlet's unit theorem, its unit group is of the form \(\{\pm 1\} \times <u>\). Notice that \(N_{\qqq / \qq}(-1) = -1\), so
        \[N_{\qqq / \qq}(-u^n) = N_{\qqq / \qq}(-1)N_{\qqq / \qq}^n(u) = -1, \quad \forall n \in \z \]
        Hence the integral solution is of the form \(u^k\) in \(\qqq\).
    \end{proof}
\end{proposition}

Notice that if \(k=0\) we can get the trival solution \((1,0)\); if \(k=-1\), we can find that \(u^{-1} = -1 + \theta\) and then get another solution \((-1,-1)\); So we need to prove that for any other \(k\), \(x-y\theta = u^k\) has no solution, one possible method is to prove that for any other \(u^k\), the cofficient with respect to base vector \(\theta^2\) is non-zero. For the case \(k>0\), by multinomial formula we can formulate
\[(1+\theta+ \theta^2)^k = \sum_{i+j+k=n} \frac{n!}{i!j!k!}\theta ^{j+2k}\]
with \(\theta ^3= 2\) we can rewrite it to get a linear combination of \(\{1,\theta,\theta ^2\}\), clearly here the cofficient of \(\theta ^2\) will not be zero so the choice of \(k\) will be limited to be less than zero. However, when \(k\leq -2\) we will find that it is difficlut to analyse, for example
\begin{align*}
    u^{-2} = v^2 &= 1- 2\theta +\theta^2 \\
    u^{-3} = v^3 &= 1+3\theta-3\theta^2 \\
    u^{-4} = v^4 &= -7-2\theta+6\theta^2\\
    &...
\end{align*}
 
The problem here is diffcult to formulate \(u^{-k}\) since there exists negative cofficient in \(v=-1+\theta\), it is not easy to deduce that whether the cofficient of \(\theta^k\) will vanish in a certain \(k\) or not, the argumment here will be not clear. Hence we need to use p-adic method. 
