\section{Field and Galois Theory}


To study the structure of the field, consider a natural map: Let \(K\) be a field
\[\phi: \zz \to K, \quad  1 \mapsto 1_K\]
It is a well-defined homomorphism, which invites a ideal of \(\zz\)
\[\ker \phi = \{n \in \zz   | n \cdot 1_k = 0_k\}\]
By the structure of \(\zz\), so there exists a integer \(p \in \zz\) such that \(\ker \phi = p \zz\),which gives the definition of the characteristic of the field.
\begin{definition}
  For any field \(K\), we define \(\mathrm{char}(K)\) be the characteristic of the field: either 0 or the smallest integer \(n \in \nn\) such that \(n\cdot1_k = 0\), by the natural map equivalently
  \[\begin{cases}
    \mathrm{char}(K) = 0 \iff  \mathrm{im}\phi \cong \zz \\
    \mathrm{char}(K) = p \iff  \mathrm{im}\phi \cong \zz/p\zz
  \end{cases}\]
\end{definition}
\textbf{The characteristic of a field is either zero or a prime number.} Suppose that \(\mathrm{char}(K) = p \neq 0\), if \(p =ab\) is not a prime, then we will get two zero divisor \(a\cdot1_K\) and \(b\cdot1_K\), but a field can not conatain any zero divisor, so \(p\) must be prime. An amazing thing in a field (or ring) with non-zero characteristic \(p\) is that we can write a equation
\[(x+y)^p = x^p + y^p\]
it has an interesting name: \textbf{freshman's dream}, this is an equality that would be written by someone who has studied very little mathematics or is just beginning to learn it.\\

Another natural map is \textbf{Frobenius endomorphism}, let \(K\) be a field with non-zero characteristic \(p\), then we define 
\[\sigma: K \to K, \quad x \mapsto x^p\]
It is a well-defined injective field homomorphism, so it is a endomorphism, but it is not necessary surjective.

\begin{example}
  Consier \(K = \ffp(x)\) a field of rational functions. \(\mathrm{char}(K) = p\) since \(\mathrm{im} \phi = \ffp\), and it is easy to observe that there exists no \(f \in \ffp(x)\) such that \(f^p(x) = x\), so the Frobenius map here is not surjective. 
\end{example}
A good case is finite field, in that case frobenius endomorphism is furthermore a automorphism, so the frobenius map will be in galois group and it reflects the structure of the finite field.

\subsection{Finite extension and splitting Field}

For the beginning, notice a classic isomorphism:
\[\rr[X]/(X-a) \cong \rr \]
For any \(a \in \rr\), and we review the isomorphism given in complex number field:
\[ \rr[X]/ (X^2+1) \cong \cc\]
Clearly, \(\cc\) is a larger field containning \(\rr\), but we wonder how we can get it from the algebra structure. Here it motivates us to consider the question from the polynoimal structure defined on a field. Firstly we need some lemmas, it is very clear after commutative ring:
\begin{lemma}
  Let \(L\) be a field containning \(K\) as a subfield, then \(L\) is a vector space over \(K\).
\end{lemma}

This lemma allows us to give a rough classification of field extension:
\begin{definition} 
  If \(L\) is a field containing \(K\) as a subfield.
  \begin{itemize}
    \item  \(L/K\) or \(K \subset L\) is denoted a \textbf{field extension} of \(K\).
    \item \(L/K\) is a \textbf{finite extension} if \(L\) is a finite-dimensional vector space over \(K\). The \textbf{degree} of the finite extension is defined as \([L:K] := \dim_K L\).
    \item \(a \in L\) is \textbf{algebraic} over \(K\) if there exists a non-zero polynomial \(f \in K[X]\) such that \(f(a) = 0\), \(L/K\) is an \textbf{algebraic extension} if every element in \(L\) is algebraic over \(K\).
    \item \(a\) is \textbf{transcendental} over \(K\) if it is not algebraic over \(K\).
  \end{itemize}
\end{definition}

Another lemma is the key to construction of field, the reason is that in a PID ring a ideal is maximal if and only if it is generated by an irreducible element.
\begin{lemma} \label{irr iff field}
  If \(K\) a field and \(p \in K[X]\), then \(p\) is irreducible if and only if \(K[X]/(p)\) is a field.
\end{lemma}


Hence the a contrsuction of field extension can be given by the quotient of polynoimal ring over a field by a good choice of ideal:
\begin{colorproposition}
  Let \(K\) be a field and \(I\) be a princiapl ideal generated by a monic irreducible polynoimal \(p \in K[X]\) of degree \(d\), Let \(L = K[X]/I\), then\\
  (1) \(L\) is a field and \(K\) can be embedded in \(L\) , so \(K\) can be identified as a subfield of \(L\).
  \\
  (2) \(p\) has a root \(\beta\) in \(L\), exactly \(\beta = X + I \in L\) \\
  (3) If \(g \in K[X]\) and \(\beta\) is a root of \(g\) in  \(L\), then \(p|g\).\\
  (4) \(p\) is the unique monic irreducible polynoimal in \(K[X]\) having \(\beta\) as a root in \(L\).\\
  (5) \(L\) can be viewed as a \(K\)-vector space with the basis \(\{1,\beta,...,\beta^{d-1}\}\).

\end{colorproposition}

  \begin{proof}
    (1) \(L\) is a field by above lemma, and we consider an embedding \(i(a) = a+I\) for any \(a \in K\), clearly it is injective and actually it is \(\pi|_K\), where \(\pi\) is the canoncial map from \(K[x]\) to \(K[x]/I\).

    (2) Suppose \(p(x) = a_0+...+a_dx^d \), then in \(L\), we have
    \begin{align*}
      p(\beta) & = a_0+a_1\beta+...+a_d\beta^d \\
      &= a_0+a_1(X+I)+...+a_d(X+I)^d \\
      &= a_0+a_1(X+I)+...+a_d(X^d+I) \\
      &= p(X)+I=I
    \end{align*}
    Here \(p(X) \in I\) and \(I\) is the zero element in \(L\).

    (3) If \(p\) does not divide \(g\), then \(gcd(p,g) = 1\) since \(p\) is irreducible, so PID ring \(K[X]\) imples the existence of \(r, t\) in \(K[X]\) such that
    \[1 = s(x)p(x)+t(x)g(x)\]
    put \(x=\beta\) in \(L\), then \(1 = 0\) leads to a contradiction.

    (4) immediately from (3). For (5) we use euclidean divison for polynoimal, any \(f \in K[X]\), there exists \(q ,r \in K[X]\) with \(\deg r < d\) such that \( f(x) = q(x)p(x)+r(x)\), then \(f+ I = r + I\) in \(L\), and if \(r(x) = b_0+...+b_kx^k\), \(k < d\), by opreations of ideal
    \[r+I = b_0 + b_1\beta+....+ b_k\beta^k\]
    so \(\{1,\beta,...,\beta^{d-1}\} \) spans \(L\). They are linearly independent because if we assume they are linearly dependent, that means there exists a polynoimal \(h \in K[X]\) of degree \(< d\) such that \(h\) has \(\beta\) as the root, but by (3) we know that \(p | h\), which leads to a contradiction since \(d= \deg p \leq \deg h \).
  \end{proof}

  Review the example in the beginning, the complex number field can also be constructed by adjoining element \(i\) to \(\rr\) such that \(i^2+1 = 0\), and we alaways write a complex number of the form \(a+ib\) with \(a,b\) real numbers, which is another construction of field extension, and more generally we have the following proposition:
\begin{proposition}
  Let \(K\) be a field and \(f \in K[X]\) irreducible, adjoin element \(c \notin K\) such that \(c\) is a root of \(f\), then \(K[c]\) is a field containning \(K\) as a subfield, and it can be written as \(K(c)\).
  \[K(c) = \{a_0+a_1c+...+a_kc^k| a_1,...,a_k \in K, k= \deg f-1\}\]
  \begin{proof}
    \(K[c]\) naturally is a ring with same identity with \(K\), so we just need to find the inverse. For any polynoimal \(p \in K[X]\) with degree less than \(\deg f\), it is co-prime with \(f\) since \(f\) is irreducible, so by Bezout's theorem for polynoimal, there exists \(r, t \in K[X]\) such that 
    \[p(x)r(x)+t(x)f(x) = 1\]
    hence we take \(x = c\) then immediately \(p(c)r(c) = 1\), so in \(K[c]\) we find the inverse of \(p(c)\). and notice that \(K[c]\) is a field garantee any rational polynoimal can have the form of polynoimal, so \(K[c] = K(c)\) evidently.
  \end{proof}
\end{proposition}

Here we find two method to extension the field, the first method is consider the quotient of polynomial ring, and we can embed \(K\) into it; the second is more direct, we just add some element in the field. 



\begin{colortheorem}[Structure of field extension]$ \\$
Let \(L/K\) be field extension of \(K\) and \(a \in L\) is algebraic, then
\\(1) There exists a unique \textbf{monic} irreducible polynoimal in \(K[X]\) having \(a\) as a root, formally we call it \textbf{minimal polynoimal} of \(a\) over \(K\), and denote it by \(\Pi_a\).
\\(2) If \(I = (\Pi_a)\), then there exists an isomorphism
\[\phi: K[X]/I \to K(a)\]
with \(X+I \mapsto a\) and \(c+I \mapsto c\) for all \(c \in K\).
\\(3) Let \(L'/K\) is another field extension and \(a' \in L'\) is also a root of \(\Pi_a\), then there exists an isomrphism \(\psi: K(a) \to K(a')\) such that \(\psi|_K = \mathrm{id}_K\) and \(\psi(a) = a'\).
\end{colortheorem}

\begin{proof}
  It needs to consider the \textbf{valuation map}, we define \(h:K[X] \to L\) by \(p \mapsto p(a)\), then clearly \(\ker h\) contains all polynoimals having \(a\) as a root, then notice that \(K[X]\) is a principal ideal ring, so \(\ker h\) must be the form \((p)\) with monic \(p \in K[X]\). By first isomorphism theorem, we know that \(K[X]/I \cong K(a)\), then by Lemma \ref{irr iff field} \(p\) must be irreducible, hence we prove (1) and (2), and we can draw commute diagram to finish (3):
\[\begin{tikzcd}
	{K[X]} && {K[a]} \\
	\\
	{K[a']} && {K[X]/I}
	\arrow["h", from=1-1, to=1-3]
	\arrow["{h'}"', from=1-1, to=3-1]
	\arrow["\pi"', from=1-1, to=3-3]
	\arrow["\simeq", from=3-3, to=1-3]
	\arrow["\simeq"', from=3-3, to=3-1]
\end{tikzcd}\]
where \(h'(p) = p(a')\), and \(\pi\) is the canoncial map. 
\end{proof}

\begin{remark}
    The theorem shows a induction
    \[\{\text{algebraic elements in } L/K\} \leadsto \{\text{monic irreducible polynoimal over } K\}\]
    conversely it is not right since the field extension may be not enough large, which refers the \textbf{"closure"} in the sense of algebraic element. For statement (3), a simple example is \(\qq(\sqrt{2}) = \qq(-\sqrt{2})\), and monic irreducible polynoimal \(X^2-2 \in \qq[X]\).
\end{remark}

\begin{colorexample}
    We consider the field extension \(\qq(\sqrt[3]{2})/\qq\), here the minimal polynoimal of \(\sqrt[3]{2}\) over \(\qq\) is \(X^3-2\), we can sovle it in \(\cc\) with roots:
    \[\sqrt[3]{2}, \quad \sqrt[3]{2}\omega, \quad \sqrt[3]{2}\omega^2 \quad  \text{with } \omega = e^{2\pi i/3}\]
    by above theorem we can connclude \(\qq(\sqrt[3]{2}) \cong \qq(\sqrt[3]{2}\omega) \cong \qq(\sqrt[3]{2}\omega^2)\), and it also implies a \textbf{question}: quotient ring such as \(\qq[X]/(X^3-2)\) can not construct a field containing all roots of irreducible polynoimal such as \(X^3-2\).\\

    But we can construct a larger field since \(\qq(\sqrt[3]{2})\) has been a field: equivalently, we can adjoin \(\omega\) to \(\qq(\sqrt[3]{2})\), or consider the irreducible polynoimal \(X^2+X+1\) over \(\qq(\sqrt[3]{2})\), then we can get a larger field \(\qq(\sqrt[3]{2},\omega)\) containing all roots of \(X^3-2\).
\end{colorexample}

For a polynoimal \(f \in K[X]\), we say \(f\) \textbf{splits} over \(K\) if there exists \(c,a_1,\dots,a_n \in K\) such that \(f\) can be factorized as linear terms:
\[f(x) = c(x-a_1)(x-a_2)\cdots(x-a_n)\]
above examples motivates us to construct a larger field containing all roots of a polynoimal:

\begin{theorem}[splitting field] $ \\$
    Let \(K\) be a field and \(f \in K[X]\) with degree \(n \geq 1\), then there exists a smallest field extension \(L/K\) such that \(f\) splits over \(L\), and it is unique up to isomorphism, formally such a field extension is called \textbf{splitting field} of \(f\) over \(K\).
\end{theorem}

\begin{proof}
    
\end{proof}






\begin{theorem}[Kronecker] If \(K\) is a field and \(f \in K[X]\), then there exists a field \(L\) containing \(K\) as a subfield and with \(f(X)\) a product of linear polynomial in \(K[X]\).
\end{theorem}

By Kronecker's theorme, we deduce a larger field where \(f\) can be decomposed completlely, so it is meaningful to define the field
\begin{definition}
  Let \(L/K\) be a field and \(f \in K[X]\).

  - \(f\) splits over \(L/K\) if there exists \(c,a_1,...a_n \in K\) such that
  \[f(x) = c(x-a_1)\cdots(x-a_n)\]
  with \(n = \deg f\).

  - \(L/K\) is called a splitting field of \(f\) over \(K\) if it is the smallest field such that \(f\) splits.
\end{definition}
  
The existence of the splitting field is ensured by above theorem, since for any \(f \in K[X]\), we can obtain a field extension \(L/K\) such that \(f\) splits, and then \(K(a_1,...,a_n) \subset L\) as a splittin field.
