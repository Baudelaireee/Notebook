\section{Toplogical Groups}

\subsection{Basic properties}
Group and Topological space is two different objects, group is algebraic structure, topological space is geometric structure, but it is not strange to combine them together.

\begin{definition}
A \textbf{topological group} is a object \((G,\mathcal{T},\cdot)\) such that
\begin{itemize}
    \item \((G,\cdot)\) is a group
    \item \((G,\mathcal{T})\) is a topological space
    \item The group structure is compatible with topological structure, i.e. multiplication 
    \[m: G \times G \to G, \quad (x,y) \mapsto x \cdot y\]
    and inverse \[i: G \to G, \quad x \mapsto x^{-1}\]
    are continuous maps.
\end{itemize} 
\end{definition}

With the basic definition, we can talk about the properties of topological groups, here is a useful lemma, it shows that the topological group is \textbf{homogeneous}, i.e. the local topology is same at any point.
\begin{lemma}
    Let \(G\) be a topological group, then for any \(g \in G\), we define \textbf{left translation} and \textbf{right translation} by
    \begin{align*}
    L_g: G \to G, \quad x \mapsto gx\\
    R_g: G \to G, \quad x \mapsto xg
    \end{align*}
    Then both \(L_g\) and \(R_g\) are homeomorphisms on \(G\).
\end{lemma}
\begin{proof}
    It is easy to verify that \(L_{gh} = L_g \circ L_h\), and \(R_{gh} = R_h \circ R_g\), then we have
    \[L_g^{-1} = L_{g^{-1}}, \quad R_{g}^{-1} = R_{g^{-1}}\]
    and we can check the continuity of \(L_g\) and \(R_g\) by composition.
\end{proof}

It is an important in the following proofs of properties, we first talk about the basic properties of topological groups.

\begin{colorproposition} \label{openiscloesd}
    Let \(G\) be a topological group, then\\

    (1) each open subgroup \(H < G\) is also closed.\\

    (2) each closed subgroup \(H < G\) with finite index is also open.

\end{colorproposition}

\begin{proof}
    The subgroup \(H\) gives a partition of \(G\) by left cosets:
    \[G = \bigsqcup_{g \in \mathcal{S}} gH\]
    with \(\mathcal{S} \subset G\) the set of representatives in \(G/H\), and we fix the representatives for \(H\) be \(e\), then
    \[G - H = \bigsqcup_{g \in S-\{e\}}gH\]
    If \(H\) is open, then by translation each \(gH\) is also open, thus the union of open sets \(G-H\) is also open, hence \(H\) is closed, which finishes that proof of (1). For the statement (2), the finite index shows that \(\mathcal{S}\) is finite, and we suppose that the index is \(n\). If \(n=1\), then \(H = G\) is open clearly; otherwise, we can deduce that \(G-H\) is the finite union of closed sets \(gH\) with \(g \in \mathcal{S}-\{e\}\), thus it is closed, it can be done similarly by translation, hence \(H\) is open sice \(G-H\) is closed.
\end{proof}


Then it is the \textbf{separation} of topological groups.
\begin{colorproposition}
    Let \(G\) be a topological groups, then the following statements are equivalent:\\

    (1) \(G\) is Hausdorff (T2).\\

    (2) The singleton \(\{e\}\) is closed in \(G\).\\

    (3) \(G\) is T1.
\end{colorproposition}

\begin{proof}
    (3) implies (2) is clear, since T1 space makes all singletons closed; (1) implies (3) is just the implication from strong to weak separation axiom.\\
    
    (2) implies (1): \(G\) is Hausdorff if and only if the diagonal set
    \[\Delta = \{(g,g) \in G \times G \mid g\in G\}\]
    is closed under the product topology, and topolgical group induces a continous map \(f(x,y) = xy^{-1}\) for any \((x,y) \in G \times G\), then \(\Delta = f^{-1}(\{e\})\) is closed by continuity.
\end{proof}

\begin{corollary}
    Let \(G\) be a topological group and \(H < G\) be a subgroup, then the quotient space \(G/H\) is \textbf{Hausdorff} if and only if \(H\) is \textbf{closed} in \(G\).
\end{corollary}

Another topological properties is about \textbf{connectness}:
\begin{colorproposition}
    Let \(G\) be a topological group, then \\

    (1) If \(G\) is connected, then any open set containing \(e\) generates \(G\).\\

    (2) The connected component \(G_0\) of identity \(e \in G\) is a \textbf{closed normal subgroup} of \(G\), and each connected component of \(G\) is homeomorphic to \(G_0\) by translation.
\end{colorproposition}

\begin{proof}
    (1) We denote \(\langle A \rangle\) be the subgroup generated by \(A \subset G\), and we take \(A\) be an open set containing \(e\), we will prove that \(\langle A \rangle\) is open and closed, by the connectness of \(G\), which implies that \(\langle A \rangle = G\). It is clear that
    \[\bigcup_{x \in \langle A \rangle} xA \subset \langle A \rangle\]
    conversely, for any \(a \in \langle A \rangle\), we have \(a = ae\) with \(e \in A\), so \(a\) is in the union such that we can replace the inculsion by equality. Notice that \(xA = L_x(A)\) is open since \(A\) is open, so \(\langle A \rangle\) is also open as the union of open sets; By the propoistion \ref{openiscloesd}, \(\langle A \rangle\) is also closed since it is open subgroup, hence we finish the proof of (1).\\

    (2) Any connected component is closed in general topology, which can be proved by contradition to \(G_0 \subsetneq \overline{G_0}\). To prove that \(G_0\) is a subgroup, we take a continous map \(f(x,y) = xy^{-1}\) on \(G \times G\), then \(f(G_0 \times G_0)\) is connected since the finite product of connected spaces is connected, and we notice that \(a = f(a,e)\) for any \(a \in G_0\), so \(G_0 \subset f(G_0 \times G_0)\) and then we can take equality by the maximality of connected component, and it shows that \(G_0\) is exactly a subgroup. 

    To prove the normality, we take the conjugation map for any \(g \in G\):
    \[\sigma_g: G \to G, \quad x \mapsto gxg^{-1}\]
    it is homeomorphism by composition \(\sigma_g = L_g \circ R_g^{-1}\). Here \(\sigma_g(e) = e\) and the image is also connected, so \(\sigma_g(G_0) \subset G_0\) by the maximality of connected component, so it must be normal.
\end{proof}

Notice that if \(H < G\) is a subgroup, then it induces a natural equivalence relation on \(G\) by left cosets:
\[x \sim y \iff x=yh, \text{ for some } h \in H\]
similarly we can define right cosets here, and then it gives a quotient space \(G/\sim\) and we denote it by \(G/H\), and it is natrually endowed with the quoitent topology with a (continous) quoitent map:
\[\pi: G \to G/H, \quad g \mapsto gH\]
\textbf{In particular}, if \(H\) is a normal subgroup, then \(G/H\) is exactly a topological group by UPQ:
\[
\begin{tikzcd}
G \times G \arrow[r, "f"] \arrow[d, "\pi \times \pi"'] 
& G \arrow[d, "\pi"] \\
G/N \times G/N \arrow[r, "f_{G/N}"] 
& G/N
\end{tikzcd}
\]
with \(f(g,h) = gh^{-1}\), so it gives the continuity of group opreation. However, it is not so easy to talk about the the topology of quoitent space \(G/H\) directly, it is better to see the quotient space as a \textbf{homogeneous space} under the action of \(G\), hence we need to talk about acion of the topological group:

\begin{definition}
    Let \(G\) be a topological group and \(X\) be a topological space, a continous action of \(G\) on \(X\) is a well-defined action in the sense of group together with a continous operation, that means there exists a continous map
    \[\phi: G \times X \to X, \quad (g,x) \mapsto g \cdot x\]
    such that for any \(g,h \in G\) and \(x \in X\), we have\\

    (1) \(e \cdot x = x\)\\

    (2) \(g \cdot (h \cdot x) = (gh) \cdot x\)
\end{definition}

In the context of toplogical groups or Lie groups, we alwasys omit "continous" when it comes to actions, it is a convention sometimes in the liter

ature. It is similar to the algebraic action, we can define \textbf{stabilizer} of a point \(x \in G\)
\[I_x = \{g \in G \mid g \cdot x = x \}\]
It is sometimes called \textbf{isotropy group} at \(x\), and it can be written as the preimage
\[I_x = \phi_x^{-1}(\{x\}), \quad \phi_x(g) = g \cdot x\]
with \(\phi_x\) a continous map by fixing variable, hence the isotropy group is always a \textbf{closed subgroup} of \(G\) if \(X\) is Hausdorff (T1 is enough, but manifold makes sense). Similarly, we can define \textbf{orbit} of \(x\) by
\[Gx = O_x = \{g \cdot x \mid g \in G\} \]
which is the image of \(\phi_x\) on the contrary, hence it gives a natural bijection

\begin{lemma}[stabilizer-orbit] $ \\$
    Let \(G\) be a compact group and \(X\) be a Hausdorff space, If \(G\) acts continously on \(X\), then for any \(x \in G\), there exists a homemomorphism \(G/I_x \cong Gx\).
\end{lemma}
\begin{proof}
    The proof is just based on the universal property of quotient space, and the stablizier-orbit theorem in group theory, we just need to check the natural bijection induced by \(\phi_x\) is indeed a homemomorphism.
\end{proof}

In particular, the orbit gives a equivalence reliation on \(X\) by
\[x \sim y \iff y = g \cdot x, \text{ for some } g \in G\]
hence it gives a quotient space denoted by \(X/G\) and we call it \textbf{orbit space}, it is natrually endowed with the quoitent topology with a (continous) quoitent map:
\[\pi: X \to X/G, x \mapsto Gx\]
the following lemma is useful and important, it shows that the topology of group action is well-behaved from the view of techinc:

\begin{colorlemma}
    The natural quotient map \(\pi: X \to G/X\) is open.
\end{colorlemma}
\begin{proof}
    Suppose that \(U\) is an open set of \(X\), then we can prove
    \[\pi^{-1}(\pi(U)) = \bigsqcup_{g \in G} gU\]
    and then by translation each \(gU\) is open, so we can finish the proof by the definition of quotient topology that \(\pi(U)\) is open if its preimage is open.
\end{proof}

Then we can conclude

\begin{colorproposition}
    Let \(G\) be a topological group and \(H\) be a subgroup, then\\

    (1) \(H\) is closed if and only if \(G/H\) is Hausdorff.\\

    (2) If \(G\) is connected, then \(G/H\) is connected.\\

    (3) If \(G/H\) and \(H\) are both connected, then \(G\) is connected.
\end{colorproposition}

\begin{proof}
    
\end{proof}


\subsection*{Neighborhood system}

As we have mentioned before, topological group is a homogeneous space, so the local topology at one point can be translated to generate the global topology, for some convenience, we always study the local topology at identity of the group.

Formally, for any group \(G\) we define the \textbf{neighborhood system} (at \(x\)) to be the collection of subsets containing \(x\), denoted by \(\mathcal{V}_x\):
\[  A \in \mathcal{V}_x \iff \quad x \in A \land A \subset G\]
Hence we can induce a topology on \(G\) by neighborhood system (i.e. determine the open voisinage of identity).
\begin{proposition}
    Let \(G\) be a group with \(\mathcal{V}_e\) as the neighborhood system at identity \(e\), if \(\mathcal{V}_e\) satisfies the following conditions:
    \begin{enumerate}
    \item For any $U,V\in\mathcal{V}_e$, there exists $W\in\mathcal{V}_e$ such that 
    \[
    W \subset U\cap V.
    \]

    \item If $a\in U\in\mathcal{V}_e$, then there exists $V\in\mathcal{V}_e$ such that 
    \[
    Va \subset U.
    \]

    \item For each $U\in\mathcal{V}_e$, there exists $V\in\mathcal{V}_e$ such that 
    \[
    V^{-1}V \subset U.
    \]

    \item For each $U\in\mathcal{V}_e$ and each $x\in G$, there exists $V\in\mathcal{V}_e$ such that 
    \[
    x^{-1} V x \subset U.
    \]
\end{enumerate}
then there exist a unique topology \(\mathcal{T}\) such that \(\mathcal{V}_{e}\) is the set of all open voisinage of \(e\), and \((G,\mathcal{T})\) is a topological group.

\begin{proof}
    The topology \(\mathcal{T}\) is uniquely defined by translation:
    \[\mathcal{B} = \{gU \mid g \in G, U \in \mathcal{V}_e\}\]
    we just need to prove that \(\mathcal{B}\) is the topology basis. Notice the condition (1) and (2) ensure that \(\mathcal{B}\) generates a topology, the condition (3) is the requirement of continuity of multiplication and inverse; finally, the condition (4) is obligator for non-abelian group to ensure the continuity.
\end{proof}
\end{proposition} 
In partiuclar, we can naturally choose subgroup as the element in the system, the condition can be simplified as following:
\begin{corollary}
    Let \(G\) be a group with \( \mathcal{U}\) as a collection of subgroups of \(G\), if \( \mathcal{U}\) satisfies the following conditions:
    \begin{enumerate}
    \item For any $U,V\in\mathcal{U}$, there exists $W\in\mathcal{U}$ such that 
    \[
    W \subset U\cap V.
    \]

    \item For each $U\in\mathcal{U}$ and each $x\in G$, there exists $V\in\mathcal{U}$ such that 
    \[
    x^{-1} V x \subset U.
    \]
\end{enumerate}
then there exist a unique topology \(\mathcal{T}\) such that \(\mathcal{U}\) is a basis of open voisinage of \(e\), and \((G,\mathcal{T})\) is a topological group.
\end{corollary} 

Here are some classic examples of topological groups constructed by neighborhood system:
\begin{example}
    We consider the usual topology on \(\qq\), then the completion of \(\qq\) is just \(\rr\), it is another method to construct real number (see in Bourkabi's book). We review that we always write the viosinage of \(0\) as the form of \((-\varepsilon, \varepsilon)\), which motivates us to consider the system of subgroup:
    \[\mathcal{U} = \{\frac{1}{n}\zz \mid n \in \nn^*\}\]
    pay attention to the choices here! We can not take \(n\zz\) to be an open set (consider it in \(\rr\), it is closed and not open). Then we can verify the topology induced by \(\mathcal{U}\) is just the subspace topology induced by \(\rr\), i.e. it induces a usual absolute value on field \(\qq\):
    \[|x|_{\infty} = \begin{cases}
        x \quad x\geq 0 \\
        -x \quad x \leq 0
    \end{cases}\]
\end{example}
It is natural to ask that can we define other topological stucture on \(\qq\) such that we can get other different completion? The answer is yes, and a classic example is the \textbf{\(p-adic\) number field.}
\begin{example}
    We fix a prime number \(p\) and then we can define a subgroup 
    \[\zz_{(p)} = \{\frac{m}{n} \mid m,n \in \zz, p \nmid n\}\]
    and we let \(\mathcal{U}= \{p^t\zz_{(p)} \mid t \in \zz\}\) to be the collection of subgroups, which generates an unique toplogy (verify!).
    Notice that we have a unique chain in the system:
    \[\dots p^2 \zz_{(p)} \subset p \zz_{(p)} \subset \zz_{(p)} \subset p^{-1} \zz_{(p)} \subset p^{-2} \zz_{(p)} \subset \dots\]
    motivates from the above example, we consider each subgroup gives a method to measure the size of an elemnt is some sense, for example we consider the following number when \(p=3\):
    \[1, \quad 4, \quad \frac{1}{2}, \quad \frac{2}{5}\]
    they are all in \(\zz_{(p)}\), and we consider the distance of them to \(0\) can be bounded by \(\alpha\) (consider (\(-\varepsilon,\varepsilon\)) gives the set of all elements with distance to \(0\) less than \(\varepsilon\)), and then we consider some other numbers:
    \[\frac{1}{3}, \quad \frac{2}{3}, \quad \frac{7}{6}\]
    they are all in \(\frac{1}{3}\zz_{(p)}\) but not in \(\zz_{(3)}\), so we can think that their distance to \(0\) is bouned by \(\beta\), and it is natural to give a realtion \(\alpha < \beta\), which motivates us to give a new method to consider the metric on \(\qq\) by conisder the local factorization of \(p\).\\

    Hence we can define something to denote \(\alpha\) and \(\beta\) just now:
    \[v_p: \zz-\{0\} \to \nn, \quad m \mapsto \max\{k : p^k | m\}  \]
    for example \(v_3(6) = 2\), \(v_3(-2) = 0\). Then the method of measuring can be extended to \(\qq\) as following:
    \[v_p: \zz-\{0\} \to \zz, \quad \frac{m}{n} \mapsto v_p(m)-v_p(n)\]
    And we make a convention that \(v_p(0) = +\infty\), it is called \textbf{p-adic valustion}. for example \(v_3(\frac{1}{3}) = -1\), \(v_3(\frac{9}{2}) = 2\). Then we can get a measure of number
    \[p^k \zz_{(p)} \iff \{x \in \qq \mid v_p(x) \geq p^k\}\]
    but a question occurs here is that \(v_3(1) = 0\) and \(v_3(\frac{1}{3}) = -1\), as what we discussed just now, actually we hope the measure of \(1\) should be less than the measure of \(\frac{1}{3}\) (here \(0>-1\)), hence the valuation \(v_p\) can not be treated as the absolute value directly, hence we modify it as following:
    \[|-|_p: \qq \to \rr_+, \quad x \mapsto \frac{1}{e^{v_p(x)}}\]
    the convention just now forces \(|0|_p = 0\), and again we can get a clear measure of numbers:
    \[x \in p^k\zz_{(p)} \iff |x|_p \leq e^{-k}\]
    hence \(|1|_3 < |\frac{1}{3}|_3\), which admits our original idea.
\end{example}

\begin{remark}
    it is just a short introduction of p-adic number from the view of topologyical group, and here are sommething to supplement if to go deeper:\\

    (1) The metric of is induced by the absolute value:
    \[d(x,y) = |x-y|_p\]
    we can verify that is a \textbf{ultrametric (non-archimedean metric)}, i.e. a metric satifying the strong triangle inequality:
    \[d(x.z) \leq \max{d(x,y),d(y,z)},\quad \forall x,y,z \in \qq\]
    which induces different analysis structure on \(\qq\).\\

    (2) Same situation with \(\qq\) equipped by the usual absolute value, \((\qq,|-|_p)\) is not complete as a metric space, we can complete it to get the \textbf{p-adic  number field} \(\qq_p\). (The completion process is same with what we do for the real number, and we can prove that the completion is unique up to isometry.)\\

    (3) By \textbf{Ostrowski's theorem}, any non-trival absolute value on \(\qq\) is equivalent to either the usual absolute value or some \(p\)-adic absolute value. In detail, if \(|-|\) is a non-trival absolute value on \(\qq\), then either there is some \(c>0\) such that
    \[|x| = |x|_{\infty}^c, \quad \forall x \in \qq\]
    or there is some prime \(p\) and some \(c>0\) such that
    \[|x| = |x|_{p}^c, \quad \forall x \in \qq\]

\end{remark}

\subsection{Homogeneous Spaces}

Review that a group action \(G \circlearrowleft X\) is \textbf{transtive} if for any \(x,y \in X\), there exists \(g \in G\) such that \(g \cdot x  = y\), i.e. there is only one orbit.

\begin{definition}
    A \textbf{homogeneous space} is a Hausdorff topological space \(X\) with a transtive continous action of a group \(G\), in particular, we call it \textbf{G-homogeneous space} to emphasize the acting group \(G\).
\end{definition}










\newpage

\subsection{The Pontryagin Duality}


We consider the category of locally compact abelian groups:
\[\cat{\mathbf{LCA}}{\text{locally compact abelian groups } (G,+)}{\text{continuous group homomorphisms }}\]

without talking about the topology of the group, we can simiarly consider the pure algebraic structure:
\[\cat{\mathbf{Ab}}{\text{abelian group } (G,+)}{\text{ group homomorphisms}}\]
Clearly it is same as category of $\mathbb{Z}$-modules, hence it is an abelian category with tools of homological algebra, so we can choose a certain objects of abelian groups to construct a duality.

\[\hat{G}: = \Hom_{\zz}(G, \rr/\zz)\]
Clearly, \(\rr/\zz\) is an injective \(\zz\)-module, hence the Hom-functor will keep exactness. In particular, it will have a good behavior when group is finite.

\begin{colorproposition}
    If \(G\) is a finite abelian group, then \(\hat{G}\) is isomorphic to \(G\).
\end{colorproposition}
\begin{proof}
    
\end{proof}

However, the duality fails when \(G\) is infinite. For example, if \(G = \zz\), then \(\hat{\zz} \cong \rr/\zz\), but if we take dual again
\[\hat{\rr/\zz} = \End_{\zz}(\rr/\zz) \cong \prod_{p}\zz_{p}\]
clearly it is larger than \(\zz\), hence it actually fails to be a duality. What we have done is to give a exact restriction on the categroy of abelian groups to make the duality work: It depends on the topology of the group, so we need to consider the topology of the dual group firstly. Formally, we define dual group as following:

\begin{definition}
    Let \(G \in \mathbf{LCA}\), we define its \textbf{dual group} as \(G^*\) or \(\hat{G}\) by
    \[G^* := \Hom_{\mathbf{LCA}}(G, \sph^1)\]
    i.e. the group of continous group homomorphism from \(G\) to the circle group (\(\sph^1 \cong \rr/\zz\)). Among \(G^*\), the elelment are called \textbf{characters} of \(G\), and we usually denote it by 
    \[\chi: G \to \sph^1\]
    with operation defined by pointwise multiplication:
    \[(\chi_1 + \chi_2)(g) := \chi_1(g) \cdot \chi_2(g)\]
\end{definition}

Generally a Hom-functor is not necessary to keep category itself, it will maps to a category of sets, so we should verify that LCA is a good condition to make the functor closed.\\

\begin{lemma}
    Let \(G \in \mathbf{LCA}\), then its dual group \(G^*\) endowed with the compact-open topology is a locally compact abelian group, in particular
    \[V(K,\varepsilon) = \{\chi \in G^* : \chi(K) \subset U_{\varepsilon}\}\]
    for any compact subset \(K \subset G\) and \(U_{\varepsilon} = \{e^{it} \mid t \in (-\varepsilon,\varepsilon)\}\), it forms a neighborhood basis of \(0 \in G^*\).
    
\end{lemma} 

\begin{proof}
    It is easy to verify that \(G^*\) is indeed a abelian group under defined opreation, to verify it is a topological group, we need to verify the family \(\mathcal{B}\) of all \(V(K,varepsilon)\) forms a neighborhood basis of identity of \(G^*\): (1) zero character \(g \mapsto 1\) is in any set clearly; (2) \(V(K,r) \cap V(L,t) = V(K\cup L, \min(r,t))\), so it is stable under finite intersection; (3) \(-V(K,\varepsilon) = V(K,\varepsilon)\) since \(U_{\varepsilon}^{-1} = U_{\varepsilon}\); (4) \(V(K,r)+V(L,t)\) is the subset of \(V(K\cup L,r+t)\), so it is stable under opreation. Hence it forms a neighborhood system of identity by (1)-(4).

    Finally, we need to verify that \(G^*\) is locally compact by Ascoli's theorem...
\end{proof}

\begin{remark}
    In particular, we denot evalution map by
    \[\ev: G \times G^* \to \sph^1, \quad (g, \chi)\mapsto \chi(g)\]
    the compact-open topology is \textbf{the coarsest topology} such that the evaluation map is continuous (page 43 [1]).
\end{remark}

A convention is that \(G^*\) always means the dual group with compact-open topology, so we finish the level of objects in the category, and then it is naturally to consider the functoriality, or the level of morphisms.

\begin{lemma}
    Let \(G,H \in \mathbf{LCA}\), and \(f: G \to H\) be a continuous group homomorphism, then it induces a natrual map
    \[f^*: H^* \to G^*, \quad \chi \mapsto \chi \circ f\]
    which is a continuous (under the compact-open topology) group homomorphism.
\end{lemma}

\begin{proof}
    For any two characters \(\chi_1, \chi_2 \in H^*\), we have
    \[f^*(\chi_1 + \chi_2)(g) = (\chi_1 + \chi_2)(f(g)) = \chi_1(f(g)) \cdot \chi_2(f(g)) = f^*(\chi_1)(g) \cdot f^*(\chi_2)(g)\]
    for any \(g \in G\), so \(f^*(\chi_1 + \chi_2) = f^*(\chi_1) + f^*(\chi_2)\) it is a group homomorphism. To prove the continuity, we take a basic open set \(V_G(K,\varepsilon)\), then
    \begin{align*}
    (f^*)^{-1}(V_G(K,\varepsilon)) &= \{\chi \in H^* \mid \chi \circ f \in V_G(K,\varepsilon)\} \\
    &= \{\chi \in H^* \mid \chi (f(K)) \subset U_{\varepsilon}\}\\
    &= V_H(f(K),\varepsilon)
    \end{align*}
    here \(f(K)\) is compact since \(f\) is continous, so the preimage of any voisinage of identity is open, hence \(f^*\) is continous by translation.
\end{proof}

It is easy to verify that \((f\circ g )^* = f^* \circ g^*\),  hence we have constructed a contravariant functor insider the category \(\mathbf{LCA}\). 

\begin{lemma}
    If \(f:G \to H\) is a surjective continuous group homomorphism, then \(f^*: H^* \to G^*\) is injective; If \(f:G \to H\) is both injective and open, then \(f^*: H^* \to G^*\) is surjective.
\end{lemma}
\begin{proof}
    If \(f\) is surjective, we take any \(\chi_1, \chi_2 \in H^*\) such that \(f^*(\chi_1) = f^*(\chi_2)\), we assume that \(\chi_1 \neq \chi_2\), then there exists \(h \in H\) such that \(\chi_1(h) \neq \chi_2(h)\), since \(f\) is surjective, there exists \(g \in G\) such that \(f(g) = h\), hence
    \(f^*(\chi_1)(g) \neq f^*(\chi_2)(g)\), so it is absurd.

    If \(f\) is injective, then injective module \(\sph^1\) allows us to extend a charcter \(\chi:G \to \sph^1\) to \(\tilde{\chi}: H \to \sph^1\) such that \(\tilde{\chi} \circ f = \chi\). We need to verify the continuity of \(\tilde{\chi}\): notice that \(f\) is open, so \(f(G)\) is the open subgroup of \(H\), hence we can prove that \(\tilde{\chi}\) is continous on \(f(G)\), then we can conclue that \(\tilde{\chi}\) is continous on \(H\) by translation (a group homomorphism is continous on some open subgroup, then the morphism is just continous).
\end{proof}

\begin{colortheorem}[Pontryagin-Van Kampen,1934] $ \\$
    Let \(G \in \mathbf{LCA}\) and \(G^*\) be its dual group, then evluation map gives a natural ismoprhism in \(\mathbf{LCA}\) by fixing one variable:
    \[\ev: G \to (G^*)^*, \quad g \mapsto \ev_g\]
    where \(\ev_g: G^* \to \sph^1\) is defined by \(\ev_g(\chi) = \chi(g)\). 
\end{colortheorem}

\begin{proof}
    
\end{proof}

Here is the slogan of the duality:
\begin{center}
    \emph{every LCA-group is the dual group of its dual group.}
\end{center}

The conclusion is similar with the finite-dimension vector space, hence it will be a strong tool to study the structure of LCA-groups and the analysis on it. In the sense of category theory, the Pontryagin duality is a contravariant equivalence with inverse itself:
\[(-)^* : \mathbf{LCA} \to ( \mathbf{LCA})^{op}\]

\subsection*{Consequences}

It is wonderful to see what we can get from the duality, we make a convention that \(G\) in this section always means a LCA-group.\\

\begin{proposition}
    In the sense of topological isomophism
    \[G_1 \oplus G_2 \cong G_1 \times G_2\]
    and 
    \[(G_1 \times G_2)^* \cong G_1^* \times G_2^*\]
\end{proposition}
\begin{proof}
    In the sense of category of abelian groups, we have
    \[(G_1 \times G_2)^* \cong G_1^* \oplus G_2^* \cong  G_1^* \times G_2^*\]
    So we just need to verify the isomophism is homemomorphism under the compact-open topology.
\end{proof}

\begin{colorproposition}
    If \(G\) is compact, then \(G^*\) is discrete; If \(G\) is discrete, then \(G^*\) is compact.
\end{colorproposition}


\begin{proof}
    
\end{proof}

\begin{colorproposition}
    If G is compact group, then\\

    - \(G\) is metrizable if and only if \(G^*\) is countable.\\

    - \(G\) is connected if and only if \(G^*\) is torsion-free.
\end{colorproposition}
